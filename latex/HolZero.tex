% =====================================================================
% HOL Manual LaTeX Source: logic
% =====================================================================

\documentclass[12pt,fleqn,oneside]{book}

\usepackage[italian]{babel}         % lingua italiana
\usepackage[latin1]{inputenc}

\usepackage{latexsym}
\usepackage{amsmath}
\usepackage{amssymb}
%\usepackage{amsbsy}
\usepackage{amsthm}

\newtheorem{theorem}{Theorem}
\newenvironment{proof1}[1][Dimostrazione]{\textbf{#1.} }{\ \rule{0.5em}{0.5em}}

\usepackage{makeidx}
\usepackage{alltt}
\usepackage{./LaTeX/layout}
\usepackage{graphicx}
\usepackage[hidelinks,hypertexnames=false]{hyperref}
\usepackage{tikz}
\usepackage{tikz-qtree}

% ---------------------------------------------------------------------
% Input defined macros and commands
% ---------------------------------------------------------------------

\input{./LaTeX/commands}
\makeindex
\begin{document}

   \setlength{\unitlength}{1mm}           % unit of length = 1mm
   \setlength{\baselineskip}{16pt}        % line spacing = 16pt

   % ---------------------------------------------------------------------
   % prelims
   % ---------------------------------------------------------------------

%   \pagenumbering{roman}                  % roman page numbers for prelims
%   \setcounter{page}{1}                   % start at page 1

   \begin{titlepage}
\null\vskip-47pt
\hbox to \textwidth {\bf [traduzione italiana] { \hfil \today}}

\setcounter{page}{1}                      % titlepage IS page 1

\vspace*{60mm}


{\fontfamily{ptm}\selectfont
\begin{center}
\scalebox{1.4}{\begin{tabular}{c} \huge Hol Zero \end{tabular}} \\
\end{center}}


%\begin{center}
%\includegraphics[width=3.8cm]{../Logo/lantern}
%\end{center}

\vfill
\end{titlepage}

% To kick a blank page with no header
\thispagestyle{empty}
\cleardoublepage

%%% Local Variables:
%%% mode: latex
%%% TeX-master: "logic"
%%% End:
                        % description title page
   \include{preface}                      % preface to entire description
   \tableofcontents                       % table of contents


   \cleardoublepage
%   \pagenumbering{arabic}                % arabic page numbers
%   \setcounter{page}{1}

   \chapter{Introduzione}


Questo � il Manuale Utente di HOL Zero e fornisce una descrizione completa di come usare il sistema.
Esso � destinato a tutti gli utenti, dai novizi agli sviluppatori esperti e agli 
auditor di dimostrazioni. Si faccia riferimento al file README nella cartella documentazione\footnote{Si tratta della cartella del programma originale in OCaml} per i 
documenti correlati, che includono un tutorial [ancora non scritto!!!] per gli utenti nuovi alla dimostrazione 
di teoremi in generale e un glossario per le definizioni precise di tutti i termini tecnici qui utilizzati.

\section{Concetti Base}

\subsection{HOL Zero}

HOL Zero � un dimostratore di teoremi HOL, cio� un programma che supporta dimostrazioni formali e lo sviluppo 
di teorie nella logica HOL (si veda pi� avanti). Si tratta di un dimostratore di teoremi relativamente semplice che si 
concentra su buone funzionalit� di base, ed � migliore rispetto ad altri sistemi HOL in aree quali la robustezza 
architetturale, lo sviluppo della sintassi concreta, un prettyprinting completo e non ambiguo, e la leggibilit� 
del codice sorgente. Queste qualit� aumentano le sue credenziali di affidabilit�, rendendolo adatto come verificatore 
di dimostrazioni per l'auditing di grandi dimostrazioni create su altri sistemi HOL (si veda pi� avanti). C'� persino 
un premio di 100 \$\footnote{Per il programma originale in OCaml} per chiunque trovi dei difetti legati alla validit� (si veda il file Bounty.txt nella cartella 
principale di HOL Zero). Le stesse qualit� lo rendono adatto come esempio pedagogico per logici, programmatori e 
informatici che vogliano comprendere come funzionano i dimostratori di teoremi.

Va comunque notato che pur essendo adatto al controllo di grandi dimostrazioni, HOL Zero non � adatto allo sviluppo 
di dimostrazioni di grandi dimensioni. Esso, infatti, supporta soltanto uno stile di dimostrazione nella semplice 
deduzione naturale, e manca di funzionalit� interattive ed automatiche avanzate che altri sistemi HOL hanno.

L'interazione utente con HOL avviene immettendo istruzioni a riga di comando in formato ASCII in una sessione di HOL 
Zero. Queste istruzioni sono di fatto espressioni nel linguaggio di programmazione ML che vengono valutate da un 
interprete ML una volta immesse. In ogni caso, non � necessario che l'utente abbia una profonda conoscenza di ML per 
usare HOL Zero.

Coloro che hanno una pi� profonda conoscenza di ML possono estendere le funzionalit� di HOL Zero. HOL Zero � implementato 
nel dialetto OCaml dell'ML, ed � possibile estendere il programma immettendo definizioni ML in una sessione di HOL Zero 
(la sessione di HOL Zero non � nient'altro che una sessione ML con il codice sorgente di HOL Zero incorporata). Qualsiasi 
di queste estensioni sono sicure nel senso che non possono introdurre incoerenze logiche nel sistema HOL Zero. Questa 
sicurezza � garantita dal fatto che HOL Zero � implementato secondo quella che viene chiamata un'architettura nello `stile LCF' 
(si veda la Sessione 4.1.3), che � usata anche in altri sistemi HOL. HOL Zero supporta lo standard Common HOL (si veda 
sotto), e questo facilita il porting del sorgente e delle dimostrazioni da altri sistemi HOL.

\subsection{La logica HOL}

La logica HOL � una logica predicativa tipizzata, classica, di ordine superiore, cio� una logica predicativa con un sistema di 
tipi, con la legge del terzo escluso come teorema, e con la possibilit� di quantificare su funzioni. E' basata sul lambda calcolo 
tipizzato di Alonzo Church. Ha un sistema polimorfico di tipi relativamente semplice che non � dipendentemente tipizzato e non 
supporta la quantificazione su variabili di tipo. Si faccia riferimento al glossario per una spiegazione estesa di questi concetti.

La logica HOL fu sviluppata per la prima volta negli anni 1980 per un sistema prototipo chiamato Cambridge HOL, ed � ora supportata 
dalla famiglia di dimostratori di teoremi HOL che include HOL4, ProofPower HOL, HOL Light � Isabelle/HOL. Questi sistemi sono 
stati utilizzati come strumenti affidabili essenziali in una variet� di progetti industriali, che includono la verifica dello sviluppo 
di microcircuiti integrati per computer e software safety-critical. Essi sono anche preminenti nella formalizzazione della matematica, 
in particolare nell'innovativo progetto Flyspeck di Tom Hales per formalizzare la sua dimostrazione della congettura di Keplero.

\subsection{Auditing di dimostrazione}

L'auditing delle dimostrazioni � il processo umano di determinare se uno script di dimostrazione che afferma di dimostrare un enunciato 
lo prova realmente. Questo dovrebbe essere fatto quando un'importante dimostrazione formale � stata completata, perch� ci sono varie 
insidie quando si usano dei dimostratori di teoremi, ognuna delle quali pu� concepibilmente rendere la pretesa dimostrazione formale 
completamente invalida. Questi variano da errori di base come il fatto che lo script di dimostrazione non risulti in una dimostrazione 
formale di successo, o il fatto che lo script dimostri l'enunciato sbagliato, a problemi pi� sottili come il fatto che le costanti che 
occorrono nell'enunciato dimostrato abbiano definizioni non corrette, che la teoria accumulata dal dimostratore di teoremi sia diventata 
incoerente, o il fatto che il dimostratore di teoremi stesso abbia degli errori nel suo disegno o nella implementazione.

L'auditing delle dimostrazioni indirizza questi problemi revisionando lo script di dimostrazione. Questo richiede come minimo il sistema 
originale per eseguire lo script, cos� che esso possa essere ri-eseguito attraverso quel sistema. L'auditor ha bisogno anche di eseguire 
controlli sull'enunciato dimostrato, le definizioni usate e la coerenza della teoria. Questi controlli sarebbero fatti normalmente usando 
il sistema originale, e l'auditor dovrebbe tener conto delle insidie in questo sistema ed assicurarsi in qualche modo che queste insidie 
non siano state sfruttate (in modo intenzionale o meno) dallo script di dimostrazione.

HOL Zero offre una soluzione pi� efficace rispetto a quella di usare il sistema originale per eseguire questi controlli, agendo come 
un controllore della dimostrazione indipendente da usare per rieseguire la dimostrazione formale originaria (usando qualche oggetto 
di importazione della dimostrazione, per esempio si veda di seguito il paragrafo Common HOL) e per eseguire i controlli su di essa. 
Questo rimuove ogni preoccupazione legata a insidie relative alla consistenza nel sistema originale e la necessit� di esaminare lo 
script di dimostrazione. A causa delle sue credenziali di affidabilit�, non c'� alcuna necessit� da parte dell'auditor di indirizzare 
problemi legati alla consistenza nel controllore di dimostrazione. Cos� il processo di auditing della dimostrazione � semplificato e 
allo stesso tempo � significativamente ampliata la sicurezza che esso fornisce.

\subsection{Common HOL}

Common HOL � uno standard per le funzionalit� di base di un sistema HOL, che ha lo scopo di facilitare la portabilit� del codice sorgente 
e delle dimostrazioni formali tra i membri della famiglia HOL. Esso consiste nelle seguenti componenti:


\begin{itemize}
	\item la specifica di una API di funzionalit� HOL di base, per permettere il porting del codice sorgente tra sistemi HOL compatibili;
	\item l'implementazione dell'API per vari sistemi HOL;
	\item la specifica di un formato di file di dimostrazione, per permettere il porting delle dimostrazioni formali tra sistemi HOL compatibili;
	\item l'implementazioni di oggetti per l'esportazione e l'importazione delle dimostrazioni tra vari sistemi HOL.
\end{itemize}

Tutte queste componenti, eccetto l'export di dimostrazione, sono liberamente disponibili sul sito Proof Technologies. I sistemi attualmente 
supportati sono HOL Zero, HOL Light e HOL90. L'export di dimostrazione � disponibile come servizio commerciale da parte di Proof Technologies, 
bench� varie dimostrazioni esportate siano liberamente disponibili sul sito. Si veda il file README di HOL Zero per il sito e i dettagli di 
contatto.

[Si noti che queste componenti non sono ancora disponibili sul sito web!!! Contattare Proof Technologies se si desidera averli.]

\subsection{Avviare HOL Zero}

Questa sezione spiega come avviare una sessione di HOL Zero. Si assume che il lettore stia utilizzando una sessione terminale su un sistema 
di tipo Unix, e che HOL Zero sia gi� stato installato (si veda il file INSTALL nella cartella Home di HOL Zero).

Una sessione di HOL Zero � avviata lanciando l'eseguibile `holzero', che � creato quando HOL Zero viene installato. Diamo due alternative 
per avviare una sessione da una finestra terminale:

\begin{enumerate}
	\item Il modo pi� semplice � lanciare HOL Zero nella finestra terminale stessa. Comunque, si noti che le funzionalit� di editing sono povere 
	a meno che si abbia installato `rlwrap' (si veda il file INSTALL). Si esegua direttamente il comando `holzero' dalla finestra terminale:

\ml{> holzero}
	
	il simbolo {\tt >} indica il prompt dei comandi della finestra terminale
	
	\item Un'alternativa migliore per una interazione sostanziale � avviare una finestra editor che mantiene delle sotto finestre per l'input ML 
	dell'utente e per i feedback della sessione ML. Un editor buono e semplice per questo scopo � Xpp (si veda il file INSTALL), che � avviato 
	dal comando:

\ml{> xpp -c holzero}

	In Xpp, la finestra in alto � per l'editing del testo e quella in basso per i feedback della sessione. Per immettere testo nella sessione, 
	si selezioni il testo e si prema Ctrl-Enter. Il testo pu� essere immesso anche linea per linea premendo Ctrl-N ad ogni linea.
	
\end{enumerate}

I primi pochi secondi di avvio di HOL Zero richiedono il build del sistema da zero. Alcune centinaia di righe di output scorrono velocemente sullo 
schermo, e quindi appare il banner di HOL Zero, che indica un build di successo. HOL Zero � quindi pronto per ricevere comandi dall'utente. Questi 
comandi sono di fatto espressioni ML, ma l'utente neofita non ha bisogno di una buona conoscenza di ML per usare HOL Zero.

\section{Panoramica di Uso}

Questa sezione fornisce una breve introduzione a semplici operazioni in HOL Zero, incluso come immettere espressioni HOL e come eseguire una semplice 
dimostrazione. Per informazioni pi� approfondite circa l'uso di HOL Zero, il lettore dovrebbe procedere nei capitoli seguenti del manuale.

\subsection{Termini, Tipi e Teoremi}

Le espressioni nel linguaggio HOL sono chiamati termini HOL. I termini sono scritti utilizzando una serie di caratteri ASCII e racchiusi tra apici 
acuti `{\tt `}', che sono chiamati {\it term quotation}. Nel momento in cui si immette un termine in una sessione di HOL Zero questo viene controllato 
e ristampato a video.

La sintassi dei termini � semplice e intuitiva, e si va incontro a vari meccanismi sintattici. Per esempio, il seguente termine significa 
`per tutti i numeri naturali $x$, $y$ e $z$, se $x$ � minore di $y$ e $y$ � minore di $z$ allora $x$ � minore di $z$' 
(il simbolo {\tt \#} indica il prompt di OCaml):

\begin{hol}\begin{verbatim}

# `!x y z. x < y /\ y < z ==> x < z`;; 
- : term = `!x y z. x < y /\ y < z ==> x < z`
						 
\end{verbatim}\end{hol}

Se si immette un termine mal formato si ricever� un messaggio di errore.

\begin{hol}\begin{verbatim}

# `x =`;;
Exception:
[HZ] SYNTAX ERROR: Unexpected end of quotation instead of RHS for infix
"=".

\end{verbatim}\end{hol}

Si noti che i messaggi specifici di HOL Zero, diversamente da quelli che deirivano dall'interprete ML 
in generale, hanno il prefisso `{\tt [HZ]}'. Questo vale per tutti i messaggi riportati da HOL Zero, inclusi messaggi di errore, warnings e 
feedback generici all'utente.

HOL � un linguaggio tipizzato, cos� ogni termine e sottotermine ha un tipo, e i termini devono essere costruiti in modo da avere un tipo 
corretto. Questo impedisce la costruzione di enunciati privi di significato come `3 � uguale a vero'.

\begin{hol}\begin{verbatim}

# `3 = true`;;
Exception:
[HZ] TYPE ERROR: Function subterm domain type incompatible with argument
subterm type.

\end{verbatim}\end{hol}

I sottotermini nei term quotation possono essere annotati per indicare il loro tipo, facendo seguire al sottotermine il simbolo di i due punti 
(`{\tt :}') e poi il suo tipo, il tutto chiuso tra parentesi. Di default, i termini sono ristampati a video con annotazioni di tipo sufficienti 
almeno ad evitare qualsiasi ambiguit� circa il tipo di ogni sottotermine.

\begin{hol}\begin{verbatim}

# `!(x:nat) (y:nat). x = y`;;
- : term = `!(x:nat) y. x = y`

\end{verbatim}\end{hol}

Il meccanismo di inferenza del tipo � usato per risolvere i tipi nei termini.Ad ogni termine inserito senza annotazioni di tipo 
sufficienti sono assegnate delle variabili di tipo numerate per tutti i tipi non determinabili.

\begin{hol}\begin{verbatim}

# `!x y. x = y`;;
- : term = `!(x:'1) y. x = y`

\end{verbatim}\end{hol}

I tipi HOL possono essere scritti fuori dal contesto di un termine usando delle {\it type quotation}. Come le term quotation, le type quotation 
sono delimitate da accenti acuti, ma iniziano con un simbolo di due punti.

\begin{hol}\begin{verbatim}

# `:nat#nat->bool`;;
- : hol_type = `:nat#nat->bool`

\end{verbatim}\end{hol}

I teoremi HOL consistono in un insieme di assunzioni e di una conclusione, con il significato che la conclusione vale assumendo che valgano le 
assunzioni. Essi sono mostrati usando il simbolo {\it turnstile} (`{\tt |-}') per separare tutte le assunzioni dalla conclusione. Si noti che 
tutti i teoremi predimostrati nel sistema base di HOL Zero non hanno alcuna assunzione.

\begin{hol}\begin{verbatim}

# excluded_middle_thm;;
- : thm = |- !p. p \/ ~ p

\end{verbatim}\end{hol}

\subsection{Dimostrazione ed Asserzione}

Le regole di inferenza della logica HOL sono implementate in HOL Zero come funzioni ML che prendono teoremi e/o termini e restituiscono 
teoremi. Un passo di dimostrazione � eseguito semplicemente valutando l'applicazione di una tale funzione.

\begin{hol}\begin{verbatim}

# assume_rule `x + y < 5`
- : thm = x + y < 5 |- x + y < 5

# spec_rule `a = 0` excluded_middle_thm;;
- : thm = |- a = 0 \/ ~ (a = 0)

\end{verbatim}\end{hol}

Le dimostrazioni sono semplicemente espressioni ML composte con applicazioni di regole di inferenza ad ogni livello.

\begin{hol}\begin{verbatim}

# deduct_antisym_rule
       (contr_rule `~ true` (assume_rule `false`))
       (eq_mp_rule (eqf_intro_rule (assume_rule `~ true`)) truth_thm);;
- : thm = |- ~ true <=> false

\end{verbatim}\end{hol}

HOL Zero viene fornito con il supporto incorporato per le seguenti teorie matematiche di base: la logica predicativa, il lambda calcolo, 
le coppie ordinate e l'aritmetica dei numeri naturali. Le sue teorie possono essere estese usando dei comandi di teoria per aggiungere 
oggetti teoria e asserire propriet� su questi oggetti.

Per esempio, il comando di definizione delle costanti introduce una nuova costante e ritorna un nuovo teorema, che afferma che il valore 
della costante � uguale a un'espressione data. Esso prende un termine di uguaglianza con la nuova costante al suo lato sinistro e il valore 
della costante al suo lato destro.

\begin{hol}\begin{verbatim}

# new_const_definition `max_height = 7`;;
- : thm = |- max_height = 7

\end{verbatim}\end{hol}

\subsection{Strumenti di HELP}

...

\section{Capitoli successivi}

I capitoli successivi e le appendici forniscono una descrizione completa di HOL Zero da un punto di vista utente.

In termini semplici, gli utenti del sistema si dividono in quattro tipi:

\begin{itemize}
	\item utenti di base, che eseguono dimostrazioni semplici senza comprendere l'ML;
	\item utenti avanzati, che eseguono dimostrazioni usando la programmazione ML;
	\item auditor, che usano HOL Zero per controllare le dimostrazioni;
	\item programmatori, che estendono le funzionalit� di HOL Zero usando la programmazione ML;
\end{itemize}

Questo manuale si rivolge a tutte e quattro le categorie. Nonostante HOL Zero sia relativamente semplice come dimostratore di teoremi HOL, 
ha tuttavia un'interfaccia utente piuttosto estesa. Comunque non � necessario per gli utenti di base e gli auditor di dimostrazione avere 
famigliarit� con ogni suo aspetto per poter essere produttivi. Di seguito spieghiamo quali parti del manuale sono rilevanti per ciascun 
tipo di utente.

I capitoli si dividono nel modo seguente:

\begin{flushleft}

\begin{tabular}{|c|c|c|c|c|l|}
	\hline
	Chapter & U1 & U2 & A & P & Description \\
	\hline
	2  &   & * & + & * &  HOL Zero's ML programming library.\\
	3  & * & * & * & * &  HOL Zero's variant of the HOL language and its language\\
     &   &   &   &   &  utilities.  Essential reading for all, except sections\\
     &   &   &   &   &  3.7 and 3.8, which can be skipped by basic users (U1).\\
	4  & * & * & * & * &  HOL Zero's variant of the HOL logic, its inference\\
	   &   &   &   &   &  rules, assertion commands and theory.  Essential\\
	   &   &   &   &   &  reading for all.\\
	\hline
\end{tabular}
		
\end{flushleft}

Le appendici forniscono materiale di riferimento dettagliato. Le appendici A forniscono dettagli sull'interfaccia ML di HOL Zero, mentre 
le appendici B forniscono elenchi delle sue teorie, e l'appendice C fornisce una grammatica HOL. Le appendici si dividono nel modo seguente:

\begin{flushleft}

\begin{tabular}{|c|c|c|c|c|l|}
	\hline
	Appendix & U1 & U2 & A & P & Description \\
	\hline
	A1  &   & * & + & * &  ML programming library functions.\\
	A2  & * & * & * & * &  Input and display commands.\\
  A3  & + & * & + & * &  HOL language utilities.\\
  A4  & * & * & * & * &  HOL theory commands.\\
	A5  & * & * &   & * &  HOL inference rules.\\
	B1  & * & * & * & * &  HOL theory listing.\\
	B2  & * & * &   & * &  HOL theorems listing.\\
	C   & * & * & * & * &  HOL language formal grammar.\\
	\hline
\end{tabular}
		
\end{flushleft}

Legenda codici utente:

\begin{itemize}
	\item U1 - Basic users
	\item U2 - Advanced users
	\item A  - Auditors
	\item P  - Programmers
	\item *  - Highly relevant
	\item +  - Background reading
\end{itemize}


   \chapter{Libreria ML}

questo capitolo non � ancora scritto
   \chapter{Il Linguaggio HOL}

Il linguaggio HOL � un potente linguaggio formale in grado di descrivere la maggior parte della matematica. 
Questo capitolo spiega il linguaggio e la versione di HOL Zero della sua sintassi attraverso le term quotation. 
Spiega inoltre varie operazioni che possono essere eseguite sull'espressioni, e come configurare 
l'input e la stampa a video.

\section{Sintassi lessicale}

Questa sezione spiega la sintassi lessicale di HOL Zero, usata sia nelle type che nelle term quotation (si vedano 
rispettivamente le Sezioni \ref{sec:regandirregnames} e \ref{sec:reservedwords}). Si veda l'Appendice C per una descrizione in grammatica 
formale della sintassi lessicale.

\subsection{Tokens}

Le type e le term quotation si compongono di una lista di token lessicali (nello stesso modo in cui il linguaggio naturale 
scritto si compone di una lista di parole e simboli di punteggiatura). I {\it token identificatori} (analoghi alle parole del 
linguaggio naturale) sono usati per riferirsi esplicitamente a entit� HOL (cio� variabili, costanti, variabili di tipo, 
e costanti di tipo). I {\it token parole riservate} (analoghi ai simboli di punteggiatura del linguaggio naturale) aiutano 
a fornire una struttura sintattica alle quotation.

Per esempio, la seguente term quotation:
\label{exmpl:termquotetoken}

\begin{hol}\begin{verbatim}

`\x. y + foo x`;;

\end{verbatim}\end{hol}

si divide in 7 token ("`{\tt $\backslash$}"', "`{\tt x}"', "`{\tt .}"', "`{\tt y}"', "`{\tt +}"', "`{\tt foo}"', "`{\tt x}"'):

\begin{hol}\begin{verbatim}

[Resword_tok "\"; Ident_tok (false, No_mark, "x"); Resword_tok ".";
   Ident_tok (false, No_mark, "y"); Ident_tok (false, No_mark, "+");
   Ident_tok (false, No_mark, "foo"); Ident_tok (false, No_mark, "x")]

\end{verbatim}\end{hol}

I token identificatori si riferiscono alle variabili "`{\tt x}"', "`{\tt y}"' e "`{\tt foo}"' e alla costante "`{\tt +}"'. Gli 
altri, "`{\tt $\backslash$}"' e "`{\tt .}"', sono token di parole riservate.

HOL � case sensitive sia per i nomi di entit� che per le parole riservate.

\subsection{Nomi regolari e irregolari}
\label{sec:regandirregnames}

Tutte le entit� HOL hanno almeno un nome, e non ci sono restrizioni sulla forma di questo nome nel linguaggio HOL. Tuttavia, 
l'identificatore che pu� essere usato nelle quotation per riferirsi alle entit� HOL dipende dal fatto che il nome sia regolare 
o irregolare. L'indentificatore per un'entit� con un nome regolare pu� essere semplicemente lo stesso nome dell'entit� (come 
per tutte le entit� nell'Esempio \ref{exmpl:termquotetoken}), mentre l'identificatore per un'entit� con un nome irregolare richiede l'uso delle 
virgolette (si veda la Sezione \ref{sec:identifierquoting}).

Ci sono tre forme di nomi regolari:

\begin{description}
	\item[Alfanumerici] Iniziano con una lettera o con "`{\tt \_}"', seguiti da zero o pi� lettere, cifre, altri simboli "`{\tt \_}"' e apici "`{\tt '}"'
	\item[Numerici] Iniziano con una cifra, seguita da zero o pi� cifre e simboli "`{\tt \_}"' mentre non possono essere immediatamente seguiti da 
	una lettera o un apice "`{\tt '}"'
	\item[Simbolici] Uno o pi� dei seguenti caratteri: "`\\\verb�! # & * + - . / ; < = > ? @ | ~ ^ [ ] \ { }�"'
\end{description}

Tutti gli altri nomi sono irregolari. Questi includono nomi con caratteri di spazio, di punteggiatura, ed ogni combinazione di caratteri alfanumerici, 
simbolici, di spazio e punteggiatura.

\begin{description}
	\item[Caratteri di spazio] : spazio, tabulazione, line feed, form feed, carriage return.
	\item[Caratteri di punteggiatura] : "`{\tt (}"' "`{\tt )}"' "`{\tt ,}"' "`{\tt :}"'
\end{description}

\subsection{Parole riservate}
\label{sec:reservedwords}

Ci sono tre tipi di token di parole riservate:

\begin{description}
	\item[Punteggiatura] : un singolo caratteri di punteggiatura (si veda la Sezione \ref{sec:regandirregnames})
	\item[Parole chiave] : un nome regolare non numerico, dal seguente insieme fisso di otto token: "`\verb�\�"' "`{\tt .}"' 
	"`{\tt and}"' "`{\tt else}"' "`{\tt if}"' "`{\tt in}"' "`{\tt let}"' "`{\tt then}"'
	\item[Parentesi di enumerazione]:
	
		\begin{itemize}
			\item un nome regolare non numerico per delimitare l'inizio o la fine di espressioni di enumerazione
			\item l'utente pu� estendere l'insieme di parentesi di enumerazione (si veda la Sezione \ref{sec:enumExpressions})
			\item nessuna parentesi di enumerazione � definita nel sistema base di HOL Zero
		\end{itemize}
	
\end{description}

Gli identificatori per entit� con nomi che sono in conflitto con parole riservate richiedono l'uso di virgolette (si veda la Sezione \ref{sec:identifierquoting}).

Si noti che il token lessicale "`{\tt =}"' � un caso speciale nella sintassi lessicale di HOL. Nonostante normalmente sia un identificatore, e sia 
classificato come identificatore dalla sintassi lessicale, � di fatto una parola chiave quando occorre come parte di una dichiarazione-let 
(si veda la Sezione \ref{sec:letExpressions}).

\subsection{Giustapposizione di token}
\label{sec:tokenjuxtaposition}

Le quotation devono essere scritte con caratteri di spazio e parentesi sufficienti per distinguere token alfanumerici/numerici adiacenti o token 
simbolici adiacenti (sia che questi token siano identificatori o parole riservate). Per esempio, in \verb�`\ ^ . ^ = 5`� (dove "`\verb�^�"' � di fatto 
una variabile), lo spazio � inserito tra "`\verb�\�"' e "`\verb�^�"', e "`\verb�^�"' e "`\verb�.�"', che sono entrambi token simbolici.

\subsection{Quoting di indentificatori}
\label{sec:identifierquoting}

Gli identificatori per entit� con nomi irregolari o nomi che sono in conflitto con parole riservate devono essere delimitati in modo speciale. 
Questo implica l'utilizzo del carattere di doppie virgolette all'inizio e alla fine del nome, come in 

\begin{hol}\begin{verbatim}

`"then" = "foo x"`

\end{verbatim}\end{hol}

che significa la variabile nominata "`{\tt then}"' � uguale alla variabile nominata "`{\tt foo x}"'. Questo � chiamato quoting.

Tutti i caratteri "`\verb�"�"', "`{\tt $\backslash$}"` in un nome tra virgolette devono essere preceduti da un backslash di escape come in \verb�`"\\ \""`� (per un carattere 
di backslash seguito da un carattere di doppie virgolette). Tutti i caratteri backquote o i caratteri non stampabili, in un nome tra virgolette, 
devono essere inseriti attraverso un backslash seguito dal loro codice ASCII a tre cifre (con degli zero iniziali per caratteri ASCII con codice 
minore di 100), come in "`\verb�\007\127�"' per il carattere ASCII "`{\tt 7}"' seguito dal carattere ASCII "`{\tt 127}"'. Questi codici ASCII a tre cifre possono essere 
usati anche per i caratteri stampabili, come in "`\verb�\111\107�"' (per "`ok"'), ma questo naturalmente non � richiesto. 

Variabili e costanti con nomi numerici devono anch'essi essere messi tra parentesi (poich� i token numerici nelle term quotation denotano numeri naturali 
- si veda la Sezione \ref{sec:numerals}). Questo non si applica a variabili di tipo e costanti di tipo con nomi numerici.

Il quoting di nomi di entit� che non richiedono il quoting stesso (cio� nomi regolari che non vanno in conflitto) � ammesso, e denota la stessa 
cosa del nome senza virgolette.

\subsection{Segni speciali}
\label{sec:specialSymbols}

Gli identificatori in HOL Zero possono essere prefissati da un carattere speciale per definire informazioni extra. Il segno "`{\tt \$}"' 
indica che l'identificatore occorre defissato (si veda la Sezione \ref{sec:defixing}), come in "`{\tt \$=}"'\footnote{In f\# il costrutto equivalente 
per defissare un operatore consiste nel metterlo tra parentesi come in "`{\tt (+) 4 5}"'.}. E i segni "`{\tt '}"' e "`{\tt \%}"' indicano 
rispettivamente che l'identificatore � una variabile di tipo o una variabile (si veda la Sezione \ref{sec:variableAndTypeVariableMarkings}), come in {\tt `:'a`} e {\tt `\%x`}.

Questi segni sono parte dello stesso token lessicale della parte principale dell'identificatore, e devono precederla immediatamente senza che intervengano 
tra essi degli spazi. Se l'identificatore il quoting (Sezione \ref{sec:identifierquoting}), allora i segni devono essere scritti fuori e immediatamente prima delle 
virgolette di apertura, come in {\tt `$\backslash$''let''`}.

Se un identificatore ha sia un segno di defixing che un segno di variabile o di variabile di tipo, allora il segno di defixing deve comparire prima, come 
in {\tt `\$\%=`}.

\section{Tipi}

I tipo HOL esistono per aiutare a rafforzare una nozione di termini ben-formati, limitando come i termini possono essere costruiti (in modo specifico, 
come possono essere costruiti i termini di applicazione di funzione). Un tipo denota un insieme non vuoto distinto di valori, e tutti i valori 
denotabili hanno precisamente un tipo. Ogni termine ha un tipo, e i possibili valori denotati da un termine sono necessariamente tutti elementi 
del suo tipo. I tipi hanno un datatype ML astratto "`{\tt hol\_type}"'.

\subsection{Sintassi dei tipi primitivi}

I tipi si suddividono in due categorie sintattiche primitive:

\begin{description}
	\item[Variabili di tipo] : 
	
		\begin{itemize}
			\item consistono di un nome. Sono scritte nella forma {\tt `:'a`} (si noti che un apice precede il nomde della variabile), 
			per qualche tipo variabile con nome "`{\tt a}"'
			\item le variabili di tipo danno una dimensione di polimorfismo a un tipo (si veda la Sezione \ref{sec:typepolimorphism})
		\end{itemize}
	
	\item[Tipi composti] : 
	
		\begin{itemize}
			\item consistono di un tipo operatore "`{\tt c}"' e di una lista di tipi parametro "`{\tt t1,t2...}"': sono scritti nella forma 
				{\tt `:c`} (si noti che nessun apice precede il nome del tipo) oppure {\tt `:(t1,t2,...)c`}, per qualche tipo costante "`{\tt c}"' 
				e una lista di tipi parametro "`{\tt t1, t2, ...}"'.
			\item questa categoria pu� essere atomica (come in "`{\tt `:c`}"') o ricorsiva (come in "`{\tt `:(t1,t2)c`}"'). Essa denota 
				un'istanza di un tipo costante, con dei tipi forniti per ciascuno dei suoi tipi parametro. Una costante di tipo fornisce 
				una struttura fondamentale ai tipi, e prende un numero fisso di tipi parametro.
		\end{itemize}
	
\end{description}

Il sistema base di HOL Zero � fornito di cinque tipi costanti preimpostati: "`{\tt bool}"', "`{\tt ind}"', "`{\tt nat}"', 
"`{\tt ->}"' e "`{\tt \#}"' (si veda la Sezione 4.5). Il linguaggio HOL pu� essere esteso aumentando la list dei tipi attraverso 
l'uso della dichiarazione di costante di tipo (Sezione 4.3.2).

\subsection{Type quotations}

I tipi sono mostrati usando una rappresentazione tramite virgolette, scritti con accenti acuti che ne delimitano l'inizio e la fine, iniziando 
con il simbolo "`:"' al loro interno (come in "`{\tt `:bool->bool`}"').

Le type quotation possono essere usate anche per digitare i tipi. Ogni possibile tipo ha sia una rappresentazione tramite virgolette che una 
rappresentazione interna.

Si noti che un altro modo per inserire tipi consiste nell'usare funzioni di costruttozione sintattica (si veda la Sezione 3.8.2).

Le Sezioni \ref{sec:fixity}, 3.6, 3.9 e 3.10 elaborano ulteriori aspetti delle type quotation.

\subsection{Tipi funzione}

Una particolare costante di tipo, chiamata {\it operatore di tipo funzione} o "`{\tt ->}"' (scritta con fixity infissa), � fondamentale nel linguaggio HOL. 
Essa prende due tipi parametro (chiamati {\it tipo dominio} e {\it tipo rango}), e denota funzioni che mappano elementi nel tipo dominio a 
elementi nel tipo rango.

Un'istanza dell'operatore di tipo funzione � conosciuta come {\it tipo funzione}. Per esempio {\tt `:nat->bool`} � un tipo funzione per funzioni 
che mappano numeri naturali a valori booleani. Il tipo di un termine di lambda astrazione � necessariamente un tipo funzione, dove il tipo dominio � 
il tipo della variabile che lega, e il tipo rango il tipo del corpo dell'astrazione.

Per essere ben formato, un termine di applicazione di funzione ha delle restrizioni sui tipi dei suoi sottotermini. Il sottotermine funzione deve 
avere un tipo funzione, e il suo tipo dominio deve essere uguale al tipo del sottotermine argomento.

Per esempio, la seguente applicazione di funzione � ben formata:

\begin{hol}\begin{verbatim}

# `(f:nat->bool) (x:nat)`;;
- : term = `(f:nat->bool) x`

\end{verbatim}\end{hol}

Mentre la seguente applicazione di funzione � mal formata:

\begin{hol}\begin{verbatim}

# `(f:bool->bool) (x:nat)`;;
Exception: [HZ] TYPE ERROR: Function subterm domain type not equal to
argument subterm type.

\end{verbatim}\end{hol}

\subsection{Polimorfismo di tipi}
\label{sec:typepolimorphism}

I tipi in HOL possono essere polimorfi, cio� un singolo tipo pu� fornire una famiglia di tipi tramite l'uso di variabili di tipo. Le 
variabili di tipo in un tipo possono poi essere istanziate per dare un tipo pi� specifico. Questo permette di evitare di replicare le teorie 
per ogni istanziazione richiesta.

Le funzioni {\tt type\_tyvars} e {\tt term\_tyvars} ritornano le variabili di tipo rispettivamente in un dato tipo e in un dato termine HOL.

\begin{hol}\begin{verbatim}

# type_tyvars `:('a#'b->bool)->('a#'b)->bool`;;
- : Type.hol_type list = [`:'a`; `:'b`]

# term_tyvars `!(x:'a). P x`;;
- : Type.hol_type list = [`:'a`]

\end{verbatim}\end{hol}

Si veda la Sezione 3.7 per i dettagli su come le variabili di tipo possono essere manipolate.

\subsection{Calcolo dei tipi}

Il tipo di ogni termine pu� essere calcolato in ultima analisi a partire dai tipi degli atomi del termine. 
Questo pu� essere fatto ricorsivamente applicando le seguenti due regole ai sottotermini fino a quando si raggiungono 
gli atomi del termine.

\begin{itemize}
	\item Il tipo di un'applicazione di funzione � il tipo rango del tipo della funzione
	\item Il tipo di una lambda astrazione � un tipo funzione, con il tipo della sua variabile legante come suo tipo dominio e il tipo 
	del corpo come suo tipo rango.
\end{itemize}

La funzione {\tt type\_of} ritorna il tipo HOL di un dato termine. Per esempio:

\begin{hol}\begin{verbatim}

# type_of `(2, true, x:'a)`;;
- : hol_type = `:nat#bool#'a`

\end{verbatim}\end{hol}

\subsection{Inferenza di tipo}

Come parte del processo di parsing di una term quotation in un termine interno, le quotation immesse sono sottoposte a un processo 
di inferenza di tipo, per determinare i tipi di ogni atomo del termine.

Per esempio, per la seguente term quotation:

\begin{hol}\begin{verbatim}

`x = y \/ y = (true,false)`

\end{verbatim}\end{hol}

l'inferenza di tipo determiner� che entrambe {\tt x} e {\tt y} sono di tipo

\begin{hol}\begin{verbatim}

`:bool#bool`

\end{verbatim}\end{hol}

L'inferenza di tipo individuer� qualunque inconsistenza di tipo all'interno della term quotation e sollever� di conseguenza un 
messaggio di errore (si veda la Sezione 3.10.3).

Alle term quotation immesse � permesso avere delle ambiguit� circa i tipi specifici dei loro sottotermini (eccetto che per le 
variabili overloaded - si veda la Sezione \ref{sec:variableOverloading}), purch� i tipi sconosciuti siano autoconsistenti. L'inferenza di tipo determina 
i tipi pi� generali possibili per i sottotermini, e alloca delle variabili di tipo numerate per i tipi non determinati.

Per esempio nel seguente sottotermine:

\begin{hol}\begin{verbatim}

`f x = (a,false)`

\end{verbatim}\end{hol}


\begin{itemize}
	\item La variabile {\tt x} � allocata al tipo {\tt `:'1`},
	\item La variabile {\tt a} � allocata al tipo {\tt `:'2`} e
	\item La variabile {\tt f} � allocata al tipo {\tt `:'1->'2\#bool`},
\end{itemize}

Per evitare ambiguit�, si pu� usare l'annotazione di tipo per dichiarare esplicitamente i tipi dei sottotermini nelle term quotation 
(Sezione 3.6.1). Si noti che un'annotazione di tipo fissa il tipo del suo sottotermine associato nella term quotation, e 
che l'inferenza di tipo deduce solo i tipi dei sottotermini che non hanno annotazioni di tipo. 

Per esempio, nel seguente sottotermine:

\begin{hol}\begin{verbatim}

`f x = (a:nat,false)`

\end{verbatim}\end{hol}


\begin{itemize}
	\item La variabile {\tt x} � allocata al tipo {\tt `:'1`},
	\item La variabile {\tt a} � allocata al tipo {\tt `:nat`} e
	\item La variabile {\tt f} � allocata al tipo {\tt `:'1->nat\#bool`},
\end{itemize}

\subsection{Overloading delle variabili}
\label{sec:variableOverloading}

In HOL � permesso l'overloading delle variabili - in altre parole possono esistere nello stesso scopo variabili con lo stesso nome 
ma con tipi differenti.

Per esempio nel seguente sottotermine ci sono due variabili distinte:

\begin{hol}\begin{verbatim}

`(x:nat) = x /\ (x:ind) = x`

\end{verbatim}\end{hol}

Le term quotation con l'overloading di variabili devono avere annotazioni di tipo sufficiente ad evitare qualsiasi ambiguit� 
(si veda la Sezione 3.6.1).

\section{Termini}

Le espressioni ben formate nel linguaggio HOL sono chiamate termini. Essi hanno un datatype ML astratto {\tt term}. 

I termini sono usati come base per rappresentare teoremi (si veda la Sezione 4.1).

\subsection{Sintassi dei termini primitivi}

Nonostante in una term quotation si possa usare una variet� di meccanismi sintattici, tutti i termini si suddividono in sole 
quattro categorie sintattiche primitive:

\begin{description}
	\item[Variabile] :
		(La forma di un termine variabile � {\tt `v`} per qualche variabile con nome "`{\tt v}"')
		Questa categoria atomica denota un'occorrenza di una data variabile. Una variabile rappresenta un valore arbitrario all'interno 
		di un termine HOL dato. Il nome della variabile pu� essere qualsiasi sequenza di caratteri ASCII, ed � case sensitive.
	\item[Costante] :
		(La forma di un termine costante � {\tt `c`} per qualche costante con nome "`{\tt c}"')
		Questa categoria atomica denota un'occorrenza di una data costante. Una costante rappresenta un valore fisso. Il nome di una 
		costante pu� essere una qualsiasi sequenza di caratteri ASCII, ed � case sensistive.
	\item[Applicazione di funzione] :
		(La forma di un termine applicazione di funzione � {\tt `f x`}, per qualche funzione "`{\tt f}"' e argomento "`{\tt x}"')
		Questa categoria ricorsiva ha un sottotermine funzione e un sottotermine argomento. Denota il valore che la funzione 
		assegna all'argomento.
	\item[Lambda astrazione] :
		(La forma di un termine lambda astrazione � {\tt `$\backslash$v. b`}, per qualche variabile "`{\tt v}"' e corpo "`{\tt b}"')
		Questa categoria ricorsiva ha un sottotermine variabile legante e un sottotermine corpo. Denota la funzione anonima che 
		assengna al suo argomento il valore determinato dal corpo, dove il corpo � espresso nei termini della variabile legante, che 
		rappresenta un segnaposto per l'argomento.
\end{description}

Queste quattro categorie primitive sono usate per la rappresentazione interna dei termini.

Tutte le term quotation possono essere espresse usando puramente le categorie primitive (usando le parentesi dove necessario per 
delimitare i sottotermini). Per esempio: 

\label{exmpl:3.3.1}
\begin{hol}\begin{verbatim}

`$! (\v1. $! (\v2. $==> ($= (f v1) (f v2)) ($= v1 v2)))`

\end{verbatim}\end{hol}

(Si veda la Sezione \ref{sec:specialSymbols} per il simbolo speciale "`\verb�$�"')

Il sistema base di HOL Zero � fornito con quarantaquattro costanti (che includono "`{\tt true}"', "`{\tt false}"', "`{\tt =}"', 
"`{\tt ==>}"', "`{\tt !}"' - si veda la Sezione 4.5). Il linguaggio HOL pu� essere esteso per averne di pi� usando la dichiarazione 
di costante (si veda la Sezione 4.3.1). 

\subsection{Term quotation}

I termini sono mostrati usando una rappresentazione tramite virgolette, scrivendoli tra accenti acuti come delimitatori 
(cio� "`{\tt `}"') all'inizio e la fine (come in {\tt `x + 5`}). La sintassi delle term quotation permette di scrivere le 
espressioni matematiche in uno stile intuitivo eppure formale.

Per esempio il seguente termine:
\label{exmpl:3.3.2}
\begin{hol}\begin{verbatim}

`!v1 v2. f v1 = f v2 ==> v1 = v2`

\end{verbatim}\end{hol}

significa "`Per ogni $v_1$ e $v_2$, se la funzione $f$ mappa $v_1$ e $v_2$ agli stessi valori allora, 
$v_1$ e $v_2$ sono uguali"'.

La forma tramite virgolette pu� essere usata anche per immetere i termini. Le term quotation immesse 
sono elaborate per costruire i "`termini interni"' (cio� la rappresentazione interna dei termini). In HOL Zero, ogni possibile 
termine ha una term quotation cos� come una rappresentazione interna, e immettere la term quotation di un termine interno che viene 
stampata a video da come risultato lo stesso termine interno.

Si noti che un modo alternativo per immettere termini consiste nell'uso delle funzioni di costruzione sintattica (si veda la 
Sezione 3.8.2).

Le Sezioni \ref{sec:specialConstants}, \ref{sec:fixity}, 3.6, 3.9 e 3.10 elaborano ulteriori aspetti delle term quotation.

\subsection{Abbreviazioni di termini}

Anche se forse non � immediatamente ovvio, la term quotation nell'Esempio \ref{exmpl:3.3.1} da esattamente lo stesso termine interno 
della term quotation nell'Esempio \ref{exmpl:3.3.2}. La sintassi non primitiva dell'Esempio \ref{exmpl:3.3.2} � spiegata nelle 
Sezioni \ref{sec:specialConstants} e \ref{sec:fixity}, ma non � altro che un'abbreviazione per la sintassi primitiva usata nell'Esempio \ref{exmpl:3.3.1}. I termini 
di default sono mostrati usando questa abbreviazione dove possibile, bench� HOL Zero abbia il comando {\tt set\_language\_level\_mode}
per mostrare soltanto la sintassi primitiva (si veda la Sezione 3.9.1).

Due forme di abbreviazione sono usate persino quando si mostra la sintassi primitiva, che diventerebbe altrimenti illeggibile a causa 
di un numero eccessivo di parentesi. 

Innanzitutto, una lambda astrazione che denoti una funzione che prende pi� argomenti pu� essere 
scritta usando una serie di variabili legate e un corpo, come in {\tt `$\backslash$x y z. body`}. Questa � un'abbreviazione per una lambda 
astrazione annidata, come in {\tt `$\backslash$x. ($\backslash$y. ($\backslash$z. body))`}.

In secondo luogo, l'applicazione di una funzione che prende pi� argomenti pu� essere scritta nella forma {\it curried}, dove la funzione 
� seguita dai suoi argomenti in serie, come in {\tt `f a b c`}. Questa � un'abbreviazione per una serie annidata di applicazioni di funzioni 
di un solo argomento, come in {\tt `((f a) b) c`}, dove la funzione pi� interna mappa il primo argomento a un valore che � esso stesso una 
funzione, e questa mappa il secondo argomento a una funzione che prende l'argomento successivo, e cos� via, con la funzione finale che 
mappa l'ultimo argomento al risultato finale. Le applicazioni delle costanti "`{\tt ==>}"' e "`{\tt =}"' nell'Esempio \ref{exmpl:3.3.1} sono tutte 
applicazioni curried, che prendono due argomenti.

\subsection{Funzioni}

Sin noti che le funzioni stesse sono valori (si pensi alle funzioni come a un mapping), e cos�, per esempio, una costante o una variabile 
possono denotare una funzione, l'applicazione di una funzione pu� essere essa stessa una funzione, e la variabile legante in una lambda 
astrazione pu� essere essa stessa una funzione.

E' questa mancanza di una fondamentale distinzione tra il trattamento delle funzioni dal trattamento dei valori che rende HOL un linguaggio 
di ordine superiore, perch� le variabili legate possono riferirsi a funzioni. Questa mancanza di distinzione significa anche che gli operatori, 
come il quantificatore universale, l'implicazione e l'uguaglianza, sono tutte costanti in HOL, perch� sono tutti valori fissati. Questo 
rende il linguaggio semplice eppure potente.

Si noti che tutte le funzioni in HOL sono totali, cos� ogni applicazione di funzione ha un valore.

\subsection{Variabili libere, legate e alfa-equivalenza}

Ogni variabile in un termine dato � o libera o legata all'interno del termine. Una variabile legata � una variabile che occorre nel suo corpo 
legato. Una variabile libera � qualsiasi variabile che occorre nel termine ma non � legata.

La funzione {\tt free\_vars} restituisce le variabili libere in un dato termine. Per esempio:

\begin{hol}\begin{verbatim}

# free_vars `\x. x = 5 \/ x = y`;;
- : term list = [`y:nat`]

\end{verbatim}\end{hol}

Lo scopo di una variabile legata � limitato al suo legame corrispondente, mentre lo scopo di una variabile libera � l'intero termine (escludendo 
per� qualsiasi sottotermine legante con una variabile legante con lo stesso nome e tipo).

Due termini sono detti essere {\it alfa-qeuivalenti} se sono equivalenti modulo i nomi di qualsiasi variabile legata nei termini. Questa � una 
nozione fondamentale di equivalenza nella logica HOL.

La funzione {\tt alpha\_eq} ritorna se i termini dati sono alfa-equivalenti. Per esempio:

\begin{hol}\begin{verbatim}

# alpha_eq `!x. x = 5 \/ x = y` `!z. z = 5 \/ z = y`;;
- : bool = true

\end{verbatim}\end{hol}

\section{Costanti particolari supportate}
\label{sec:specialConstants}

Certe costanti hanno una loro sintassi specifica nelle term quotation, per migliorare la leggibilit�. Questa sezione tratta queste 
costanti.

\subsection{Espressioni condizionali}

Un'espressione condizionale denota il valore di uno di due sottotermini, chiamati {\it rami}, dove la scelta � determinata da un'ulteriore 
sottotermine, con valore booleano, chiamato {\it condizione}. Il primo ramo, chiamato {\it ramo-allora}, � scelto se la condizione � 
vera, e il secondo, chiamato {\it ramo-altrimenti}, � scelto se la condizione � falsa.

Le espressioni condizionali sono scritte con {\tt if}, seguito dalla condizione, seguito da {\tt then}, seguito dal ramo-allora, seguito 
da {\tt else}, seguito dal ramo-altrimenti, come in:

\begin{hol}\begin{verbatim}

`if c then t1 else t2`

\end{verbatim}\end{hol}

({\tt c} ha tipo {\tt :bool}, mentre {\tt t1} ha lo stesso tipo di {\tt t2}) che significa "`il valore $t_1$, se $c$ ha valore 
{\it vero}, e il valore $t_2$, se $c$ ha il valore {\it falso}"'. 

Comunque, questa � di fatto un'{\it abbreviazione} per un'applicazione della costante {\tt COND}:

\begin{hol}\begin{verbatim}

`COND c t1 t2`

\end{verbatim}\end{hol}

\subsection{Espressioni-LET}
\label{sec:letExpressions}

Un'espressione-let � usata per introdurre variabili locali con il loro valore assegnato a un sottotermine. Ciascuna parte 
che introduce una variabile e il suo valore assegnato � chiamata {\it let-binding}, e il sottotermine a cui questo si 
applica � chiamato {\it let-body}. Ogni let-binding � scritto come una variabile, seguita da "`{\tt =}"', seguita dal 
valore assegnato. Per esempio:

\begin{hol}\begin{verbatim}

`let v1 = t1 and v2 = t2 and ... and vn = tn in t0`

\end{verbatim}\end{hol}

che significa "`il valore $t_0$, ma con la variabile $v_1$ rimpiazzata con il termine $t_1$, e la variabile $v_2$ rimpiazzata 
con il termine $t_2$, e poi ... e poi la variabile $v_n$ rimpiazzata con il termine $t_n$"'. Questo ha lo stesso valore 
di 

\begin{hol}\begin{verbatim}

`(\v1 v2 ... vn . t0) t1 t2 ... tn`

\end{verbatim}\end{hol}

Questa � un'abbreviazione per un'applicazione composta della costante {\tt LET}:

\begin{hol}\begin{verbatim}

`LET (... (LET (LET (\v1 v2 ... vn . t0) t1) t2)...) tn`

\end{verbatim}\end{hol}

Si noti che il simbolo "`{\tt =}"' usato in un let-binding, per separare una variabile dal suo valore assegnato, � di fatto 
un token parola riservata bench� abbia uno stato di token identificatore durante il parsing. Questa � una peculiarit� 
della sintassi lessicale di HOL.

\subsection{Espressioni coppia}

Le espressioni coppia sono scritte con parentesi aperte e chiuse, con le componenti separate da virgola, come in:

\begin{hol}\begin{verbatim}

`(t1, t2, ..., tn)`

\end{verbatim}\end{hol}

che significa un $n$-upla, con le componenti $t_1, t_2, ..., t_n$.

Questa � un'abbreviazione per l'applicazione composta della costante {\tt PAIR}:

\begin{hol}\begin{verbatim}

`PAIR t1 (PAIR t2 (... tn)...)`

\end{verbatim}\end{hol}

\subsection{Espressioni Enumerazione}
\label{sec:enumExpressions}

Le enumerazioni danno un elenco seriale di elementi con la stessa struttura, con dei delimitatori all'inizio e alla fine che indicano 
la forma particolare della struttura. Gli elementi in un'enumerazione sono separati da "`{\tt ;}"'\footnote{Nel manuale utente originale 
sembra un errore l'indicazione della "`{\tt ,}"' come delimitatore di separazione per gli elementi di una lista.}.

Per esempio, se devono essere definite delle liste, una lista di tre elementi pu� essere scritta nel seguente modo:

\begin{hol}\begin{verbatim}

`[ a; b; c ]`

\end{verbatim}\end{hol}

Il fatto che sia una lista verrebbe indicato dall'uso di \verb�[� e \verb�]� come delimitari di apertura e chiusura. Questa sarebbe 
un'abbreviazione per un'applicazione composta dell'operatore di inserimento di elementi in una lista, {\tt CONS}, alla 
costante per la lista vuota {\tt NIL}:

\begin{hol}\begin{verbatim}

`CONS a (CONS b (CONS c NIL))`

\end{verbatim}\end{hol}

Questa forma di enumerazione pu� essere scritta per termini di una data struttura se i delimitatori sono stati associati 
con l'operatore di inserimento di elementi della struttura e la costante di struttura vuota. Questo � fatto usando il comando 
"`{\tt set\_enum\_brackets}"', che prende come argomenti una coppia per i nomi delle costanti di inserimento e di lista vuota e una 
coppia per i delimitatori di apertura e chiusura. 

Per esempio, le parentesi per le liste sarebbero settate da:

\begin{hol}\begin{verbatim}

# set_enum_brackets ("CONS","NIL") ("[","]");;
[HZ] Setting "[" and "]" as enumeration brackets for constructor "CONS"
with zero "NIL".
- : unit = ()

\end{verbatim}\end{hol}

I nomi forniti per le parentesi di enumerazione devono essere regolari non numerici, cio� o simbolici o alfanumerici 
(si veda la Sezione \ref{sec:regandirregnames}), e una volta associati, questi nomi diventano parole riservate.

L'informazione circa le parentesi di enumerazione gi� settate � fornita dal comando "`{\tt get\_all\_enum\_info}"'\footnote{Nota bene: 
come indicato nella Sezione \ref{sec:reservedwords} nessuna parentesi di enumerazione � definita nel sistema base di HOL Zero, quindi 
il comando indicato restituir� una lista vuota all'avvio di HOL Zero.}. 
Per esempio:

\begin{hol}\begin{verbatim}

# get_all_enum_info ();;
- : ((string * string) * (string * string)) list =
[(("CONS", "NIL"), ("[", "]"));
 (("SERIAL_STMT", "NULL_STMT"), ("begin", "end"))]

\end{verbatim}\end{hol}

\subsection{Numerali}
\label{sec:numerals}

I numerali rappresentano valori di numeri naturali, e semplicemente prendono la forma di un nome numerico (come in {\tt `19`}). 
Qualsiasi "`{\tt \_}"' in un nome numerico � usato solamente per imbottire, e non ha alcun effetto sul valore (cos� {\tt `65\_535`} 
rappresenta lo stesso termine interno di {\tt `65535`}).

Un numerale in HOL Zero non � una costante, ma � di fatto soltanto l'abbreviazione per una serie di applicazioni delle funzioni 
{\tt BIT0} e {\tt BIT1} alla costante {\tt ZERO}, con un'etichettatura esterna attraverso la funzione {\tt NUMERAL}. 
Per esempio {\tt `19`} � un'abbreviazione per la seguente espressione:

\label{exmpl:3.4}
\begin{hol}\begin{verbatim}

`NUMERAL (BIT1 (BIT1 (BIT0 (BIT0 (BIT1 ZERO)))))`

\end{verbatim}\end{hol}

La costante {\tt ZERO} rappresenta il numero naturale $0$. La funzione {\tt BIT0} raddoppia il suo argomento, e la 
funzione {\tt BIT1} raddoppia il suo argomento ed aggiunge $1$. La funzione {\tt NUMERAL} semplicemente ritorna il 
suo argomento\footnote{In formato numerico a base 10.}, ed � usata per etichettare un numerale atomico. Usare le operazioni composte 
{\tt BIT0} e {\tt BIT1} su {\tt ZERO} corrisponde direttamente alla notazione binaria, con {\tt BIT0} per 
0-bits e {\tt BIT1} per 1-bits, ma con l'ordine dei bit invertito. Cos� l'esempio \ref{exmpl:3.4} rappresenta $10011$ (base 2), 
cio� $19$ (base 10). Si noti che i {\tt BIT0} applicati direttamente a {\tt ZERO} non hanno effetto sul valore. Termini interni con {\tt BIT0}
applicati direttamente a {\tt ZERO} non sono nemmeno stampati a video come numerali, per assicurare che i numerali abbiano 
una rappresentazione interna univoca.

\section{Fixity}
\label{sec:fixity}

Agli identificatori di variabile, costante e tipo costante pu� essere assegnato uno status sintattico speciale, chiamato 
{\it fixity}\index{fixity}, pemettendo loro di essere scritti prima, dopo o in mezzo, ai loro argomenti. La fixity � 
un elemento decisamente importante per aumentare la leggibilit� delle term quotation.

\subsection{Fixity di termine e fixity di tipo}
\label{sec:termAndTypeFixity}

La {\it fixity di termine}\index{fixity di termine} si pu� riferire alla fixity degli identificatori di variabili e di costanti. 
Qualsiasi variabile o costante che condivida lo stesso nome deve avere la stessa fixity di termine.

Le fixity di termine disponibili sono: nonfix, prefix, infix, postfix, binder.

La {\it fixity di tipo}\index{fixity di tipo} si riferisce alla fixity di tipi costanti, ed � indipendente dalla fixity di termini 
(e cos�, per esempio, lo stesso identificatore, potrebbe avere fixity di tipo infissa e fixity di termine prefissa).

Le fixity di tipo disponibili sono: nonfix, infix.

\subsection{Fixity Nonfix}
\label{sec:nonfix}

La fixity di default di un identificatore � {\it nonfix}\index{nonfix}. Nella fixity di termine, nonfix � per le funzioni che 
sono applicate nella classica forma curried (come in {\tt `P x y z`} che occorrono con o senza argomenti, o per costanti che 
non sono funzioni (come in {\tt `x`}).

Nella fixity di tipo, nonfix � per i tipi operatore che appaiono dopo i loro tipi argomento (come in {\tt `:('a,'b)map`}) e 
per i tipi base (come in {\tt `:bool`}). Si noti che bench� i tipi operatore appaiano dopo i loro tipi parametro, essi hanno 
fixity nonfix e non postfix.

\subsection{Fixity prefix}
\label{sec:prefix}

La fixity {\it prefix}\index{prefix} � per termini operatori unari che appaiono prima dei loro argomenti (come in \verb�`~ true`�). 
Diversamene dagli operatori nonfix, non sono richieste parentesi per le applicazioni composte di un operatore prefisso (cos� 
\verb�`~ ~ true`� � lo stesso di \verb�`~ (~ true)`�).

\subsection{Fixity infix}
\label{sec:infix}

La fixity {\it infix}\index{infix} � per operatori binari o tipi operatore che appiaono in mezzo ai loro argomenti o i loro 
tipi parametro (come in {\tt `x + y`} o {\tt `:'a -> 'b`}).

Per ridurre il numero di parentesi richieste quando si scrivono espressioni infisse composte, le fixity infix hanno un 
{\it numero di precedenza}\index{numero di precedenza} e un {\it lato di associazione}\index{lato di associazione}. Gli operatori 
con un numero di precedenza pi� alto legano in modo pi� stretto, e cos� le espressioni con operatore di precedenza pi� alto che 
sono sottoespressioni di espressioni con operatore di precedenza pi� basso non devono essere racchiuse tra parentesi. Il 
lato di associazione ({\it associazione a sinistra}\index{associazione a sinistra}, {\it associazione a destra}\index{associazione a destra}, 
{\it non associazione}\index{non associazione}) indica come un'espressione composta senza parentesi sullo stesso operatore si 
scompone, dove le parentesi per operatori associativi a sinistra o a destra si raggruppano implicitamente sul lato di associazione, e 
le espressioni composte su operatori non associativi sono sempre scritte con parentesi esplicite.

Per esempio, {\tt +} � associativo a sinistra, e {\tt *} ha precedenza pi� alta di {\tt +}, cos� il seguente termine:

\begin{hol}\begin{verbatim}

`(a + b * c + d + e) * f`

\end{verbatim}\end{hol}

rappresenta lo stesso termine interno di 

\begin{hol}\begin{verbatim}

`(((a + (b * c)) + d) + e) * f`

\end{verbatim}\end{hol}

\subsection{Fixity postfix}
\label{sec:postfix}

La fixity {\it postfix}\index{postfix} � per termini operatore unari che compaiono dopo il loro argomento (come in \verb�`5 !!`�, per 
qualche operatore postfisso {\tt !!}). Le applicazioni composte di un operatore postfisso richiedono l'uso di parentesi come in 
\verb�`(5 !!) !!`�).

\subsection{Fixity binder}
\label{sec:binder}

La fixity {\it binder}\index{binder} � per i {\it quantificatori}\index{quantificatori}, come {\tt !} (per ogni, $\forall$), {\tt ?} (esiste, $\exists$) e 
{\tt \@} (seleziona), che prendono come argomento una lambda astrazione e cos� legano una variabile legante. Questi sono scritti allo stesso modo 
delle lambda astrazioni (come in {\tt `!x. x > 5`} che significa "`$\forall x, x > 5$"', ed � di fatto un'abbreviazione per 
{\tt `\$! ($\backslash$x. x > 5)`}). Come per le lambda astrazioni, il binding di una serie di variabili con lo stesso operatore pu� essere scritto 
come un singolo binding (come in {\tt `!x y. x = y`}.

\subsection{Potenza dei binding}
\label{sec:bindingPower}

Il seguente elenco sintetizza quanto strettamente le fixity legano. I primi elementi dell'elenco legano pi� strettamente dei successivi:

\begin{itemize}
	\item nonfix
	\item postfix
	\item prefix
	\item infix
	\item binder
\end{itemize}

I sottotermini che legano pi� strettamente non hanno bisogno di parentesi, e i termini sono sempre mostrati senza parentesi.

Per esempio:

\begin{hol}\begin{verbatim}

`P 0 /\ (!x. ~ P (x + 1) ==> ~ P x) ==> (!x. P x)`

\end{verbatim}\end{hol}

� lo stesso di 

\begin{hol}\begin{verbatim}

`(P 0 /\ (!x. (~ (P (x + 1))) ==> (~ (P x)))) ==> (!x. (P x))`

\end{verbatim}\end{hol}

\subsection{Defixing}
\label{sec:defixing}

La fixity di un identificatore pu� essere rilasciata localmente ({\it defixing}\index{defixing}) in una data quotation, per dare 
all'occorrenza di quell'operatore uno stato nonfix. Questo si ottiene anteponendo il simbolo "`{\tt \$}"' all'operatore, senza 
spazi di separazione, come in {\tt \$=}.

L'uso del defixing � necessario quando si fa riferimento a un operatore che non � applicato al numero completo di argomenti 
richiesto dalla sua fixity.

Si veda la Sezione \ref{sec:specialSymbols} per maggiori dettagli circa la sintassi lessicale dei simboli di defixing.

\subsection{Impostazione e verifica delle fixity}
\label{sec:settingAndQueryingFixity}

Il datatype ML {\tt fixity} per rappresentare la fixity. Esso ha le seguenti classi:

\begin{itemize}
	\item Nonfix
	\item Prefix
	\item Infix (int * assochand)
	\item Postfix
	\item Binder
\end{itemize}

dove {\tt assochand} pu� assumere i valori {\tt LeftAssoc}, {\tt RightAssoc} or {\tt NonAssoc}.

Le fixity degli identificatori di termine e di tipo possono essere impostate rispettivamente con i comandi 
{\tt set\_fixity} e {\tt set\_type\_fixity}. Essi prendono come argomenti il nome dell'identificatore e la nuova 
fixity. Per esempio:

\begin{hol}\begin{verbatim}

# set_fixity ("##", Infix (200,RightAssoc));;
[HZ] Setting fixity for name "##".

\end{verbatim}\end{hol}

Le fixity di termine e di tipo possono essere verificate usando, rispettivamente, i comandi {\tt get\_fixity} e {\tt get\_type\_fixity}. 
Per esempio:

\begin{hol}\begin{verbatim}

# get_type_fixity "->";;
- : fixity = Infix (5, RightAssoc)

\end{verbatim}\end{hol}

La fixity di un identificatore di termine o di tipo pu� essere rilasciata (globalmente) usando, rispettivamente, i comandi 
{\tt reset\_fixity} e {\tt reset\_type\_fixity}. Per esempio:

\begin{hol}\begin{verbatim}

# reset_type_fixity "x";;
[HZ] Resetting type fixity for name "x".

\end{verbatim}\end{hol}

\section{Annotazioni}
\label{sec:annotations}

Nelle quotation possono essere usate tre forme di annotazione. Lo scopo delle annotazioni � quello di aggiungere informazioni chiare 
circa una quotation e cos� evitare ambiguit�. HOL Zero assicura che le quotation siano sempre mostrate con un livello sufficiente di 
annotazioni per evitare qualsiasi ambiguit�.

\subsection{Annotazioni di tipo}
\label{sec:typeAnnotations}

Ai sottotermini possono essere aggiunte delle annotazioni di tipo circondando il sottotermine con delle parentesi e facendo seguire al 
sottotermine il carattere {\tt :} e quindi il suo tipo (come in {\tt `(f:'a->'b) x`}). Si noti che le parentesi possono essere omesse 
per delle annotazioni di tipo inserite al livello superiore (come in {\tt `x:bool`}), o per elementi di enumerazione (come in 
{\tt `[a:nat,b]`}) o per elementi di coppie (come in {\tt `(a:nat,b:bool)`}).

Le quotation di termini e di teoremi stampati a video hanno i loro termini atomo con annotazioni di tipo sufficienti per rimuovere 
qualsiasi ambiguit� circa i tipi specifici dei sottotermini. Per esempio:

\begin{hol}\begin{verbatim}

# `f x`;;
- : term = `(f:'1->'2) x`

\end{verbatim}\end{hol}

Di default, le annotazioni di tipo automatiche di termini e teoremi stampati a video sono sufficienti per rimuovere qualsiasi ambiguit� 
a un livello minimo. Comunque, il comando {\tt set\_type\_annotation\_mode} pu� essere usato per far s� che ogni atomo sia stampato 
con un'annotazione di tipo (si veda la Sezione 3.9.2).

Alle type quotation immesse possono essere aggiunte delle annotazioni di tipo per vincolare i tipi dei sottotermini. L'inferenza di tipo 
altrimenti assegna il tipo pi� generale per ogni termine di una quotation immessa (si veda la Sessione 3.2.6). L'uso di un'annotazione 
di tipo fissa il tipo dei suoi sottotermini associati nella term quotation, e l'inferenza di tipo deduce solo il tipo dei sotto termini 
senza annotazioni di tipo.

Le term quotation immesse normalmente non richiedono alcuna annotazione di tipo per essere accettate dal parser (purch� siano ben 
formate). Comunque, le term quotation che coinvolgono l'overloading delle variabili hanno bisogno di annotazioni di tipo sufficienti 
ad assicurare che i tipi delle variabili overloaded visibili al livello superiore del termine o di un sottotermine lambda astrazione 
siano completamente risolti a quel punto (senza prendere in considerazione l'informazione contestuale dall'esterno della lambda 
astrazione).

Per esempio, la seguente � rigettata a causa dell'ambiguit� della sua variabile di binding (che potrebbe essere di tipo {\tt `:'a`} o 
di tipo {\tt `:'b`}):

\begin{hol}\begin{verbatim}

# `x = (x:'a) /\ !x. x=(x:'a) \/ (x:'b)=x`;;
Exception: [HZ] TYPE ERROR: Overloaded var "x" type not resolved at name
closure.

\end{verbatim}\end{hol}

La seguente, invece, non ha ambiguit� (la variabile libera {\tt x} ha tipo {\tt `:'b`}) ed � accettata:

\begin{hol}\begin{verbatim}

# `x = x /\ !(x:'a). x=(x:'a) \/ (x:'b)=x`;;
- : term = `x = x /\ (!(x:'a). (x:'a) = x \/ (x:'b) = x)`

\end{verbatim}\end{hol}

anche la seguente � accettata (tutte le {\tt x} all'interno della lambda astrazione sono assunte riferirsi alla stessa variabile):

\begin{hol}\begin{verbatim}

# `x = (x:'b) /\ !x. x=(x:'a) \/ x=x`;;
- : term = `(x:'b) = x /\ (!(x:'a). x = x \/ x = x)`

\end{verbatim}\end{hol}

Si noti che un identificatore con annotazione di tipo mantiene la sua fixity anche se essa � nonfix (come in 
{\tt `x (=:'a->'a->bool) y`}).

\subsection{Segno di identificazione di variabile e di variabile di tipo}
\label{sec:variableAndTypeVariableMarkings}

Gli identificatori di variabili nelle term quotation possono essere distinte dagli identificatori di costanti con lo stesso 
nome annotandoli con un segno {\tt \%} prima del nome, senza alcuno spazio di separazione (come in {\tt `\%x`}). Si noti che 
non � obbligatorio che una tale variabile abbia lo stesso tipo della costante con lo stesso nome.

Per esempio, la seguente denota una variabile con nome "`{\tt true}"' (in contrapposizione alla costante booleana "`{\tt true}"'):

\begin{hol}\begin{verbatim}

# `%true`;;
- : term = `%true:'1`;;

\end{verbatim}\end{hol}

Analogamente, gli identificatori delle variabili di tipo nelle type quotation e nelle annotazioni di tipo possono essere distinte 
dalle costanti di tipo facendo precedere al nome il simbolo "`{\tt '}"' (come in {\tt `:'bool`} per denotare una variabile di tipo 
con nome "`{\tt bool}"').

Nelle term quotation, gli identificatori con il nome di una costante senza annotazioni si assumono riferirsi alla costante. Analogamente, 
nelle type quotation e nelle annotazioni di tipo, gli identificatori senza annotazioni con il nome di una costante di tipo si assumono 
riferirsi alla costante di tipo.

Le variabili che non sono sottoposte a overload rispetto alle costanti sono stampate a video di default senza il segno di variabili, mentre tutte le variabili 
di tipo sono stampate a video di default con un segno di variabile di tipo. Questo pu� essere modificato usando i comandi 
{\tt set\_var\_marking\_mode} e {\tt set\_tyvar\_marking\_mode} (si veda la Sezione 3.9.3).

Si noti che qualsiasi identificatore di variabile marcato con "`{\tt \%}"' avr� la stessa fixity degli identificatori di costante con lo stesso 
nome (a meno che l'identificatore sia defissato usando "`{\tt \$}"' - si veda la Sezione 3.5.8).

Si vedano la Sezione per maggiori dettagli sulla sintassi lessicale dei segni di variabile e di variabile di tipo.

\section{Instanziazione e matching}

... instanziazione

\begin{hol}\begin{verbatim}

# type_inst : (hol_type * hol_type) list -> hol_type -> hol_type

# tyvar_inst : (hol_type * hol_type) list -> term -> term

# var_inst : (term * term) list -> term -> term

\end{verbatim}\end{hol}

\section{Funzioni sintattiche}

Le funzioni sintattiche sono funzioni ML di utilit� dedicate a una particolare categoria sintattica di tipo o termine, che agiscono 
al livello superiore di un'espressione di tipo o di termine. Ci sono tre tipologie di funzioni sintattiche: decostruttori, costruttori e 
discriminatori. Tutti le tipologie sono definite per ciascuna delle categorie sintattiche primitive di tipi e termini, cos� come per 
la maggior parte delle costanti di funzione e di tipi operatori definite nel sistema HOL Zero di base.

Si veda l'Appendice A3 per una descrizione completa di ciascuna funzione sintattica.

\subsection{Decostruttori sintattici}

Le funzioni di decustruzione della sintassi servono a suddividere un tipo HOL o un'espressione di termine che ha una particolare 
categoria sintattica al suo livello superiore, ritornando le sottocomponenti dell'espressione. Il decustruttore di base per una 
data categoria sintattica ha un nome ML della forma "`{\tt dest\_<categoria>}"' (si veda la Sezione 3.8.4 per una lista completa dei 
nomi di categoria).

Per esempio, il decostruttore {\tt dest\_conj} suddivide termini congiunzione:

\begin{hol}\begin{verbatim}

# dest_conj `x = 5 /\ y = 3`;;
- : term * term = (`x = 5`, `y = 3`)

\end{verbatim}\end{hol}

Se il livello superiore dell'espressione fornita non si conforma alla categoria sintattica data � lanciata un'eccezione ML di tipo 
HolFail. Per esempio:

\begin{hol}\begin{verbatim}

# dest_conj `x = 5 \/ (y = 3 /\ z = 4)`;;
Exception: [HZ] FAIL: dest_conj - Not a conjunction.

\end{verbatim}\end{hol}

I decostruttori sintattici sono definiti per vari tipi di espressioni composte. Queste hanno un nome ML della forma 
"`{\tt strip\_<category>}"'. Per esempio:

\begin{hol}\begin{verbatim}

# strip_conj `x = 5 /\ y = 3 /\ z = 4`;;
- : term list = [`x = 5`; `y = 3`; `z = 4`]

\end{verbatim}\end{hol}

Le espressioni possono anche essere suddivise nei termini di una categoria sintattica primitiva.

Per esempio, nella sintassi primitiva, il livello superiore di un termine congiunzione � un'applicazione di funzione. 
Cos� la congiunzione di sopra pu� essere suddivisa anche nel modo seguente:

\begin{hol}\begin{verbatim}

# dest_comb `x = 5 /\ y = 3`;;
- : term * term = (`$/\ (x = 5)`, `y = 3`)

\end{verbatim}\end{hol}

Per i termini � fornito un decostruttore generico, chiamato {\tt dest\_term}, che suddivide un termine in 
una delle quattro categorie sintattiche primitive di termine. Per esempio:

\begin{hol}\begin{verbatim}

# dest_term `\x. x = 5`;;
- : destructed_term = Term_abs (`x:nat`, `x = 5`)

# dest_term `x = 5`;;
- : destructed_term = Term_comb (`$= (x:nat)`, `5`)

\end{verbatim}\end{hol}

Analogamente, � fornito un decostruttore generico per i tipi, chiamato {\tt dest\_type}, per le due categorie sintattiche 
primitive di tipo. Per esempio:

\begin{hol}\begin{verbatim}

# dest_type `:'a->'b`;;
- : destructed_type = Type_comp ("->", [`:'a`; `:'b`])

\end{verbatim}\end{hol}

\section{Costruttori sintattici}

Le funzioni di costruzione sintattica servono per creare un'espressione di tipo o di termine dalle sue sottocomponenti. 
Queste rappresentano delle alternative alle quotation di tipo e di termine (si vedano le Sezioni 3.2.2 e 3.3.2) per 
creare tipi e termini interni. Esse prendono una o pi� sottoespressioni come argomenti e ritornano un'espressione. Esse 
prendono una o pi� sottoespressioni come argomenti e ritornano un'espressione. Le funzioni costruttori di base hanno 
un nome ML della forma "`{\tt mk\_<category>}"' (si veda la Sezione 3.8.4 per una lista completa dei nomi di categoria).

Per esempio, il costruttore {\tt mk\_eq} crea un termine uguaglianza:

\begin{hol}\begin{verbatim}

# mk_eq (`x:nat`,`5`);;
- : term = `x = 5`

\end{verbatim}\end{hol}

I costruttori tipicamente hanno restrizioni sulla forma dei loro argomenti, per assicurare che essi possano creare 
soltanto espressioni ben formate. Gli argomenti che non si adeguano alle restrizioni risultano in un'eccezione ML di tipo 
HolFail. Per esempio:

\begin{hol}\begin{verbatim}

# mk_eq (`true`,`5`);;
Exception: [HZ] FAIL: mk_eq - Arg 1 type not equal to Arg 2 type.

\end{verbatim}\end{hol}

Sono anche definiti costruttori sintattici per creare espressioni composte. Questi hanno un nome ML della forma 
"`{\tt ist\_mk\_<categroy>}"'. Per esempio:

\begin{hol}\begin{verbatim}

# list_mk_forall ([`x:nat`;`y:nat`], `x = 0 \/ y < x`);;
- : term = `!x y. x = 0 \/ y < x`;;

\end{verbatim}\end{hol}

\section{Discriminatori sintattici}

I discriminatori sintattici sono funzioni ML per testare che il livello superiore di un'espressione si conformi ad una particolare 
categoria sintattica. Essi semplicemente ritornano {\tt true} se l'espressione si conforma e {\tt false} altrimenti. I discriminatori 
hanno un nome ML della forma {\tt is\_<category>} (si veda la Sezione 3.8.4 per una lista completa di nomi di categoria). Si noti che 
i discriminatori non sollevano eccezioni.

Per esempio, {\tt is\_fun\_type} � il discriminatore per i tipi funzione:

\begin{hol}\begin{verbatim}

# is_fun_type `:A->B->C`;;
- : bool = true

\end{verbatim}\end{hol}

\section{Nomi di categorie sintattiche}

La parte del nome di una funzione sintattica usata per riferirsi a una particolare categoria sintattica � un nome minuscolo alfabetico 
per la categoria.

Le categorie di termini HOL supportate dalle funzioni sintattiche, insieme con i loro nomi alfabetici sono come segue:

\begin{description}
	\item[variabile] : {\tt var} 
	\item[costante] : {\tt const}
	\item[applicazione di funzione] : {\tt comb}
	\item[lambda astrazione] : {\tt abs}
\end{description}

\begin{flushleft}

	\begin{tabular}{|l|l|l|}
		\hline
		Simbolo Matematico					&		ASCII  						&		Funzione Sintattica 		\\
		\hline		
		$=$ 												&		\verb�=�  				&		{\tt eq} 								\\
		$\wedge$  									&		\verb�/\�  				&		{\tt conj}							\\		
		$\vee$ 											&		\verb�\/�  				&		{\tt disj}							\\
		$\rightarrow$ 							&		\verb�==>�  			&		{\tt imp} 							\\ 
		$\neg$  										&		\verb�~�  				&		{\tt not} 							\\
		$\forall$ 									&		\verb�!�  				&		{\tt forall}						\\
		$\exists$ 									&		\verb�?�  				&		{\tt exists}						\\
																&		\verb�@�  				&		{\tt select}						\\
																&		\verb�COND�* 			&		{\tt if}								\\
																&		\verb�LET�* 			&		{\tt let}						    \\
																&		\verb�PAIR�*			&		{\tt pair}							\\
																&		\verb�FST�  			&		{\tt fst}							  \\
																&		\verb�SND�  			&		{\tt snd}								\\
																&		\verb�SUC�  			&		{\tt suc}								\\
																&		\verb�PRE�  			&		{\tt pre}								\\
		$+$ 												&		\verb�+�  				&		{\tt add}						    \\
		$*$ 												&		\verb�*�  				&		{\tt mult}						  \\
		$-$ 												&		\verb�-�  				&		{\tt sub}						    \\
																&		\verb�EXP�  			&		{\tt exp}						  	\\
																&		\verb�EVEN�  			&		{\tt even}						  \\
																&		\verb�ODD�  			&		{\tt mult}						  \\
		$<$ 												&		\verb�<�  				&		{\tt let}						    \\
		$\leq$ 											&		\verb�<=�  				&		{\tt le}						    \\
		$>$ 												&		\verb�>�  				&		{\tt gt}						    \\
		$\geq$ 											&		\verb�>=�  				&		{\tt ge}						    \\
		\hline
	\end{tabular}

\end{flushleft}


\begin{description}
	\item[*] Si noti che queste sono costanti supportate in modo speciale, che hanno la loro sintassi nelle term quotation (si veda la Sezione 3.4)
\end{description}

Le categorie di tipo supportate dall funzioni sintattiche, insieme con i loro nomi alfabetici, sono le seguenti:


\begin{flushleft}
	\begin{tabular}{|l|l|l|}
		\hline
		Categoria							&		ASCII 	&		Funzione Sintattica 		\\
		\hline		
		variabile di tipo 		&						&		{\tt var\_type} 				\\
		tipo composto					& 					&		{\tt comp\_type}				\\		
		tipo funzione 				& {\tt ->}	&		{\tt fun\_type}					\\
		tipo prodotto 				& {\tt \# }	&		{\tt prod\_type} 				\\ 
		\hline
	\end{tabular}
\end{flushleft}

\section{Modalit� di stampa a video}

I dettagli di come le type, term e theorem quotation sono stampate a video sono determinati dalle modalit� di visualizzazione. 
Si noti che queste modalit� non fanno differenza circa la forma delle quotation accettate come input.

\subsection{Visualizzazione della sintassi primitiva dei termini}

I termini e i teoremi possono essere mostrati a video puramente in termini delle categorie sintattiche primitive di termine 
(si vedano le Sezioni 3.3.1 e 3.3.3), impostando il livello di visualizzazione a {\tt Minimal} attraverso la funzione 
{\tt set\_language\_level\_mode}.

Visualizzare soltanto la sintassi primitiva dei termini pu� aiutare l'utente a comprendere la struttura di una term 
quotation, e pu� essere particolarmente utile per i novizi. Le costanti supportate in modo speciale (si veda la Sezione 3.4) 
sono stampate a video senza la loro propria sintassi, e i termini operatore con una fixity (si veda la Sezione 3.5) sono 
stampati defissati.

Per esempio:

\begin{hol}\begin{verbatim}

# `!n. f n = if n > 1 then (n,m) else (m,n)`;;
- : term = `!n. f n = (if n > 1 then (n,m) else (m,n))`

# set_language_level_mode Minimal;;
[HZ] Setting language level mode to Minimal.
- : unit = ()

# `!n. f n = if n > 1 then (n,m) else (m,n)`;;
- : term =
`$! (\n. $= (f n) (COND ($> n (BIT1 ZERO)) (PAIR n m) (PAIR m n)))`

\end{verbatim}\end{hol}

(Si veda la Sezione \ref{sec:specialSymbols} per il simbolo speciale "`\verb�$�"')

Il livello normale di visualizzazione della sintassi dei termini pu� essere ripristinata impostando 
nuovamente il livello su {\tt Full}:

\begin{hol}\begin{verbatim}

# set_language_level_mode Full;;
[HZ] Setting language level mode to Full.
- : unit = ()

\end{verbatim}\end{hol}

Si noti, che quando si mostra la sintassi primitiva dei termini, tutti i tipi sono ancora stampati nel modo normale 
(cio� con le fixity delle costanti di tipo che sono state impostate).

\section{Modalit� di annotazione di tipo}

I termini e i teoremi in HOL Zero sono sempre mostrati con annotazioni di tipo sufficienti ad evitare qualsiasi ambiguit� 
circa i tipi di ogni sottotermine (si veda la Sezione 3.6.1). Di default, questo � fatto a un livello minimo, cos� che non 
ci siano pi� annotazioni di quelle necessarie ad evitare le ambiguit�. Comunque, tutte le variabili e tutte le costanti 
polimorfiche possono avere i tipi annotati impostando il livello di annotazione di tipo a {\tt Full}.

Per esempio:

\begin{hol}\begin{verbatim}

# `f x = x + 2`;;
- : term = `f x = x + 2`

# set_type_annotation_mode Full;;
[HZ] Setting type annotation mode to Full.
- : unit = ()

# `f x = x + 2`;;
- : term = `(f:nat->nat) (x:nat) (=:nat->nat->bool) (x:nat) + 2`

\end{verbatim}\end{hol}

La modalit� normale di a visualizzazione delle annotazioni di tipo pu� essere ripristinata impostando di nuovo 
il livello su {\tt Minimal}.

\subsection{Modalit� di visualizzazione dei segni di variabile e di variabile di tipo}

I tipi, i termini e i teoremi in HOL Zero sono sempre mostrati con sufficienti segndi di variabile e di variabile di 
tipo per evitare qualsiasi ambiguit� circa la distinzione tra variabili e costanti e tra variabili di tipo e 
costanti di tipo (si veda la Sezione 3.6.2). Di default, tutte le variabili di tipo sono etichettate come tali, 
e i termini variabili sono etichettati solo se il loro nome � in overlaod con un'altra costante nella teoria. Questo 
pu� essere modificato impostando le modalit� per i segni di variabile e variabile di tipo.

Per esempio:

\begin{hol}\begin{verbatim}

# `!x. (P x <=> true) \/ (P x <=> false)`;;
- : term = `!(x:'1). (P x <=> true) \/ (P x <=> false)`

# set_var_marking_mode Full;;
[HZ] Setting variable marking mode to Full.
- : unit = ()

# set_tyvar_marking_mode Minimal;;
[HZ] Setting type variable marking mode to Minimal.
- : unit = ()

# `!x. (P x <=> true) \/ (P x <=> false)`;;
- : term = `!(%x:1). (%P %x <=> true) \/ (%P %x <=> false)`

\end{verbatim}\end{hol}


\subsection{Sintesi delle modalit� di visualizzazione}

Le impostazioni delle modalit� di visualizzazione correnti sono sintetizzate eseguendo il comando 
{\tt show\_display\_modes}.

Per esempio:

\begin{hol}\begin{verbatim}

# show_display_modes ();;
HOL Zero Display Modes
	 Type annotation mode :        Minimal
	 Variable marking mode :       Minimal
	 Type variable marking mode :  Full
	 Language level mode :         Full
- : unit = ()

\end{verbatim}\end{hol}

\section{Messaggi di errore nel parsing delle quotation}

Immettere una type o una term quotation mal formata risulter� in un messaggio di errore del parser piuttosto che in un 
tipo o termine interno. I messaggi di errore del parser sono sollevati come eccezioni ML di tipo HolErr.

Gli errori di parsing si dividono in tre categorie (corrispondenti ai tre passi principali di parsing): errori lessicali, 
errori sintattici, errori di tipo. Gli errori lessicali sono riconosciuti prima di tutti gli errori sintattici, i quali 
sono riconosciuti prima di qualsiasi errore di tipo. Soltanto un errore di parser (corrispondente al primo errore individuato, 
leggendo da sinistra a destra) � mostrato per ogni quotation. Un messaggio di errore di parsing inizia con la sua categoria 
in lettere maiuscole, seguita da una descrizione dettagliata dell'errore.

\subsection{Errori lessicali}

Gli errori lessicali riguardano token mal formati (si veda la Sezione 3.1) nelle type e nelle term quotation. Questi 
possono riguardare:

Problemi circa l'uso dei segni di defixing/variabile/variabile di tipo:
	
\begin{hol}\begin{verbatim}

# `$ = a b`;;
Exception: [HZ] LEXICAL ERROR: Defix mark ($) must immediately precede
name, without space.

\end{verbatim}\end{hol}

Caratteri di escape sbagliati all'interno delle virgolette:
	
\begin{hol}\begin{verbatim}

# `f "\744"`;;
Exception: [HZ] LEXICAL ERROR: Character escape code out of range - must
be 000 to 255.

\end{verbatim}\end{hol}

Caratteri alfabetici all'interno di token numerici:

\begin{hol}\begin{verbatim}

# `23x`;;
Exception: [HZ] LEXICAL ERROR: Non-numeric character in numeric token.

\end{verbatim}\end{hol}

Caratteri non stampabili:

\begin{hol}\begin{verbatim}

# `.`;;           
Exception: [HZ] LEXICAL ERROR: Unprintable ASCII character 183 - use ASCII
escape code inside quotes.

\end{verbatim}\end{hol}

[ dove {\tt .} rappresenta il carattere non stampabile con codice ASCII 183 ]

\subsection{Errori sintattici}

Gli errori sintattici riguardano problemi su come i token sono combinati nelle type e nelle term quotation. 
Questi possono riguardare:

Delimitatori di apertura (come "`{\tt (}"') che non hanno il corrispondente di chiusura:

\begin{hol}\begin{verbatim}

# `x = (3 + y * 4`;;
Exception: [HZ] SYNTAX ERROR: Opening "(" but no subsequent ")".

\end{verbatim}\end{hol}

Un'occorrenza anticipata di una parola riservata, con la conseguente mancanza di un sottotermine:

\begin{hol}\begin{verbatim}

# `(a,b,,d)`;;
Exception: [HZ] SYNTAX ERROR: Missing pair component.

\end{verbatim}\end{hol}

L'occorrenza inaspettata di un token quando ne � richiesto un altro:

\begin{hol}\begin{verbatim}

# `let x = 5 else in x + y`;;
Exception: [HZ] SYNTAX ERROR: Unexpected reserved word "else" instead of
"and" or "in".

\end{verbatim}\end{hol}

L'occorrenza inaspettata di un token dopo un sottotermine superiore sintatticamente corretto:

\begin{hol}\begin{verbatim}

# `x = 5 else`;;
Exception: [HZ] SYNTAX ERROR: Unexpected reserved word "else" after
syntactically-correct leading subterm.

\end{verbatim}\end{hol}

La fine inaspettata della quotation:

\begin{hol}\begin{verbatim}

# `if (P x) then Q x else`;;
Exception: [HZ] SYNTAX ERROR: Unexpected end of quotation instead of
conditional else-branch.

\end{verbatim}\end{hol}

Un token marcato come variabile di tipo all'interno di un termine, o un token marcato come variabile all'interno 
di un tipo:

\begin{hol}\begin{verbatim}

# `p <=> 'a \/ b`;;
Exception: [HZ] SYNTAX ERROR: Type variable "a" encountered outside type
annotation.

\end{verbatim}\end{hol}

Problemi relativi all'uso di identificatori con fixity:

\begin{hol}\begin{verbatim}

# `= x`;;
Exception: [HZ] SYNTAX ERROR: Missing LHS for infix "=".

\end{verbatim}\end{hol}

Una quotation vuota:

\begin{hol}\begin{verbatim}

# `:`;;
Exception: [HZ] SYNTAX ERROR: Empty type quotation.

\end{verbatim}\end{hol}

\subsection{Errori di tipo}

Gli errori di tipo riguardano problemi nell'inferenza di tipi coerenti per i termini atomo nelle term quotation, e con tipi 
mal formati forniti nelle type e nelle term quotation. In modo specifico, questi riguardano:

Liste di parametri forniti ai tipi costanti con un'ariet� errata:

\begin{hol}\begin{verbatim}

# `:nat bool`;;
Exception: [HZ] TYPE ERROR: Type constant "bool" has arity 0 but is used
with type param list length 1.

\end{verbatim}\end{hol}

Sottotermini funzione che non hanno un tipo funzione dedotto:

\begin{hol}\begin{verbatim}

# `(f:'a) x = 5`;;
Exception: [HZ] TYPE ERROR: Function subterm does not have function type.

\end{verbatim}\end{hol}

Sottotermini funzione e argomento con tipi dedotti incompatibili

\begin{hol}\begin{verbatim}

# `true + 1`;;
Exception: [HZ] TYPE ERROR: Function subterm domain type incompatible with
argument subterm type.

\end{verbatim}\end{hol}

Annotazioni di tipo incompatibili con il tipo dedotto per i loro sottotermini:

\begin{hol}\begin{verbatim}

# `(3:bool) = 4`;;
Exception: [HZ] TYPE ERROR: Subterm type incompatible with type annotation.

\end{verbatim}\end{hol}

Tipi dedotti per le variabili che danno origine a un overloading delle variabili ambiguo:

\begin{hol}\begin{verbatim}

# `!x. (x:'a)=x /\ (x:'b)=x`;;
Exception: [HZ] TYPE ERROR: Overloaded var "x" type not resolved at name
closure.

\end{verbatim}\end{hol}


   \chapter{La logica HOL}

La logica HOL � lo strumento per eseguire deduzioni nel linguaggio HOL. Questo capitolo descrive 
la variante di HOL Zero della logica, e come usarla per costruire teorie e dimostrare teoremi.

\section{Teoremi}

I teoremi sono asserzioni che si � stabilito valere nella logica, o attraverso una dimostrazione 
(si veda la Sezione 4.2) o attraverso un'asserzione. Essi hanno il datatype ML astratto {\tt thm}.

Un teorema prende la forma di un insieme di assunzioni e una conclusione, ognuna delle quali sono 
termini HOL booleani. Il significato di enunciati di questo tipo � che si pu� dimostrare che la 
conclusione vale se valgono le assunzioni. Si noti che i teoremi non hanno mai assunzioni differenti 
ma alfa equivalenti.

\subsection{Stampa a video di un teorema}

I teoremi sono scritti con un simbolo "`{\tt |-}"' che separa le assunzioni dalla conclusione, con le 
assunzioni separate dal simbolo "`{\tt ,}"' (come in {\tt x = 5, y > 3 |- x * y > 15}. Si noti che, dal 
momento che HOL Zero utilizza un'architettura di tipo LCF (si veda la Sezione 4.1.3), questo modo di 
scrivere i teoremi pu� essere usato solo per stamparli a video e non per immetterli nel sistema.

Un'altra conseguenza dell'architettura di tipo LCF l'etichettatura con il datatype ML {\tt thm}, mostrata 
dall'interprete ML come parte della visualizzazione di un teorema, diventa un segno dell'autenticit� che 
un teorema � stato realmente stabilito valere nella logica (o per asserzione o per dimostrazione). 
Per esempio:

\begin{hol}\begin{verbatim}

# excluded_middle_thm;;
- : thm = |- !p. p \/ ~ p

\end{verbatim}\end{hol}

I modi di visualizzazione (si veda la Sezione 3.9) influiscono sul modo di visualizzare i teoremi allo 
stesso modo in cui influiscono sulla visualizzazione delle term quotation.

\subsection{Funzioni di utilit� per i teoremi}

La funzione di utilit� {\tt dest\_thm} suddivide un dato teorema nelle sue assunzioni e nella sua conclusione. 
L'utilit� {\tt concl} restituisce soltanto la conclusione del teorema, e {\tt asms} soltanto le sue 
assunzioni. Si noti che le assunzioni sono restituite come una lista, nonostante l'ordine degli elementi 
nella lista sia irrilevante e non ci possano essere elementi ripetuti.

\begin{hol}\begin{verbatim}

# dest_thm excluded_middle_thm;;
- : term list * term = ([], `!p. p \/ ~ p`)

# concl excluded_middle_thm;;
- : term = `!p. p \/ ~ p`

\end{verbatim}\end{hol}

L'utilit� {\tt thm\_free\_vars} restituisce tutte le variabili libere di un dato teorema.

\begin{hol}\begin{verbatim}

# let th1 = trans_rule (assume_rule `f = (\x.x+y)`)
											 (assume_rule `(\x.x+y) = g`);;
val th1 : thm = (\x. x + y) = g, f = (\x. x + y) |- f = g

# thm_free_vars th1;;
- : term list = [`g:nat->nat`; `f:nat->nat`; `y:nat`]

\end{verbatim}\end{hol}

L'utilit� {\tt thm\_alpha\_eq} restituisce se due dati teoremi sono alfa-equivalenti. L'alfa-equivalenza di 
teoremi � definita sulla base dell'alfa-equivalenza dei termini (si veda la Sezione 3.3.5). Perch� due 
termini siano alfa-equivalenti, � necessario che la loro conclusione sia alfa-equivalente, e che gli insiemi delle 
loro assunzioni abbiano la stessa dimensione e per ogni elemento in un insieme ci sia un elemento alfa-equivalente 
nell'altro insieme.

\begin{hol}\begin{verbatim}

# thm_alpha_eq (assume_rule `!(x:'a).x=x`) (assume_rule `!(y:'a).y=y`);;
- : bool = true

\end{verbatim}\end{hol}

\subsection{Costruzione di teoremi e architettura LCF}

I teoremi possono essere costruiti in HOL Zero usando le sue regole di inferenza (si veda la Sezione 4.2) e/o 
i comandi di asserzione (si veda la Sezione 4.4), oppure attraverso estensioni da parte dell'utente al codice 
sorgente ML.

HOL Zero � implementato secondo un'architettura di tipo LCF. Questo significa che, qualunque meccanismo sia 
usato per costruire un teorema, � in un'ultima analisi fatto in termini delle regole primitive d'inferenza 
(si veda la Sezione 4.2.2) e/o i comandi primitivi di asserzione (si veda la Sezione 4.4.7) della logica. 
Questo si ottiene facendo del datatype ML per la rappresentazione interna dei teoremi un datatype astratto, con 
le regole primitive di inferenza e i comandi primitivi di asserzione come suoi unici costruttori.

Grazie a questa architettura di tipo LCF, qualsiasi preoccupazione circa la validit� della deduzione in HOL Zero 
� limitata alla correttezza del disegno e dell'implementazione delle primitive. Questo significa che gli utenti 
possono implementare con sicurezza delle estensioni al sistema senza compromettere la validit� del sistema 
stesso.

\subsection{Archiviazione dei teoremi}

Esistono due strumenti per salvare teoremi con un nome come indice. Il primo � per salvare i risultati di un teorema 
per un uso generale, e richiede che i suoi teoremi non abbiano variabili libere n� assunzioni (assicurando in questo 
modo che essi soddisfino gli standard minimi per catturare in modo pulito una propriet� generale). Tali teoremi 
sono archiviati usando il comando {\tt save\_thm}.

\begin{hol}\begin{verbatim}

# save_thm ("sub_def_thm2", conjunct2_rule subtract_def);;
[HZ] Storing theorem "sub_def_thm2".
- : thm = |- !m n. m - SUC n = PRE (m - n)

\end{verbatim}\end{hol}

I teoremi generali archiviati possono essere estratti usando il comando {\tt get\_thm}. Il comando 
{\tt get\_all\_thms} restituisce tutti i teoremi generali archiviati. Il sistema base di HOL Zero � fornito con 
oltre 130 teoremi generali archiviati (si veda l'Appendice B2). 

\begin{hol}\begin{verbatim}

# get_thm "excluded_middle_thm";;
- : thm = |- !p. p \/ ~ p

# get_all_thms ();;
- : (string * thm) list =
[("bool_cases_thm", |- !p. (p <=> true) \/ (p <=> false));
 ("conj_absorb_disj_thm", |- !p q. p /\ (p \/ q) <=> p);
 ...]

\end{verbatim}\end{hol}

Il secondo strumento di archiviazione di teoremi � per archiviare risultati di teoremi intermedi (lemmi) che non sono 
in una forma ideale per un uso generale. Questo non ha restrizioni sulla forma dei suoi teoremi. I lemmi sono salvati 
usando il comando {\tt save\_lemma}. Come per i teoremi generali, c'� un comando per restituire un dato lemma archiviato 
({\tt get\_lemma}) e un comando per restituire tutti i lemmi archiviati ({\tt get\_all\_lemmas}).

\begin{hol}\begin{verbatim}

# save_lemma ("temp_result_1", assume_rule `l < m /\ m < SUC n`);;
[HZ] Storing lemma "temp_result_1".
- : thm = l < m /\ m < SUC n |- l < m /\ m < SUC n

\end{verbatim}\end{hol}

\section{Dimostrazione formale}

Una dimostrazione formale � il processo di usare un meccanismo logico formale di deduzione per stabilire che un enunciato 
vale nella logica. Lo scopo principale di un dimostratore di teoremi � quello di supportare questo processo. In HOL Zero, 
la dimostrazione � esguita valutando l'applicazione di funzioni ML chiamate {\it regole d'inferenza}, tipicamente 
in modo annidato, e tipicamente facendo riferimento a teoremi stabiliti in precedenza.

Per esempio, la seguente espressione ML � una dimostrazione di un risultato di base nella logica proposizionale (classica):

\begin{hol}\begin{verbatim}

# deduct_antisym_rule
        (contr_rule `~ true` (assume_rule `false`))
        (eq_mp_rule
          (eqf_intro_rule (assume_rule `~ true`))
          truth_thm );;
    - : thm = |- ~ true <=> false

\end{verbatim}\end{hol}

Essa usa 5 regole d'inferenza ({\tt deduct\_antisym\_rule}, {\tt contr\_rule}, {\tt assume\_rule}, {\tt eq\_mp\_rule} and {\tt eqf\_intro\_rule}) 
e un teorema stabilito in precedenza ({\tt truth\_thm}).

Si possono trovare dei buoni esempi di dimostrazioni reali nelle parti del codice sorgente di HOL Zero dove esso stabilisce 
la sua libreria di teoremi archiviati. Questa si estende in vari file di codice sorgente, che includono {\it boolalg.ml}, {\it boolclass.ml}, 
{\it pair.ml}, {\it nat.ml}, {\it natarith.ml} e {\it natrel.ml}. Tutte le dimostrazioni nel codice sorgente sono formattate con un'identazione 
per riflettere la loro struttura, e annotate di commenti per mostrare i risultati dei teoremi intermedi.

\subsection{Regole d'inferenza}

Una regola d'inferenza � una funzione ML che prende un teorema e/o un termine come argomenti e restituisce un teorema, calcolato 
eseguendo una deduzione sugli argomenti, in base alle regole della logica HOL. Le regole d'inferenza in ultima analisi sono basate 
sulla forma sintattica dei loro argomenti, e le regole base d'inferenza, che trattano soltanto una specifica forma sintattica, possono 
essere sintetizzate in modo elegante usando una descrizione formale della regola nello stile di Gentzen.

Per esempio, la regola {\tt eq\_imp\_rule2} prende un singolo teorema come argomento, dove l'argomento deve avere un'equivalenza logica 
come sua conclusione. Il risultato � un teorema che afferma che il lato destro dell'equivalenza implica il suo lato sinistro.

Questo � sintetizzato nella seguente descrizione nello stile di Gentzen.

\begin{prooftree}
\def\fCenter{\ \vdash\ }
\AxiomC{$A \vdash p \Leftrightarrow q$}
\UnaryInfC{$A \vdash q \Rightarrow p$}
\end{prooftree}

Qui c'� un esempio del suo utilizzo:

\begin{hol}\begin{verbatim}

# eq_imp_rule2 not_true_thm;;
- : thm = |- false ==> ~ true

\end{verbatim}\end{hol}

Una conversione � una speciale forma di regola di inferenza che prende un termine come argomento e restituisce un teorema 
con un'eguaglianza come sua conclusione, dove il lato sinistro dell'eguaglianza � il termine fornito come argomento.

Per esempio, {\tt eval\_add\_conv} � una conversione che prende un termine nella forma dell'operatore {\tt +} applicato 
a due numerali di numeri naturali, e restituisce un teorema che da il suo valore numerale.

Essa � sintetizzata dalla seguente descrizione nello stile di Gentzen:

\begin{prooftree}
\def\fCenter{\ \vdash\ }
\AxiomC{\`{}$m + n$\`{}}
\UnaryInfC{$ \vdash m + n = z$}
\end{prooftree}

Qui c'� un esempio del suo utilizzo:

\begin{hol}\begin{verbatim}

# eval_add_conv `28 + 12`;;
- : thm = |- 28 + 12 = 40

\end{verbatim}\end{hol}

Il sistema HOL Zero di base include circa 90 regole d'inferenza. Si veda l'Appendice A5 per una descrizione informale e una 
descrizione nello stile di Gentzen di ciascuna regola.

\subsection{Regole primitive d'inferenza}

Tutte le regole d'inferenza in HOL Zero sono implementate (o {\it derivate}) in termini delle altre regole d'inferenza pi� di base. 
In fondo a questa gerarchia di dipendenze ci sono 10 regole d'inferenza primitive che non sono implementate in termini di altre 
regole, e che in ultima analisi formano la base di tutte le altre regole. Queste sono chiamate le {\it regole di inferenza primitive} 
della logica. La chiamata a una regola d'inferenza derivata tipicamente risulter� in alcune chiamate a regole d'inferenza primitive.

Di seguito sono elencate le descrizioni nello stile di Gentzen delle 10 regole primitive:

%\begin{prooftree}
%\def\labelSpacing{10 pt}
%\AxiomC{\`{}$t$\`{}}
%\RightLabel{refl\_conv}
%\UnaryInfC{$\vdash t = t$}
%\end{prooftree}

\begin{prooftree}
\def\labelSpacing{10 pt}
\AxiomC{$t$}
\RightLabel{refl\_conv}
\UnaryInfC{$\vdash t = t$}
\end{prooftree}

\begin{prooftree}
\def\labelSpacing{10 pt}
\AxiomC{$(\lambda x.\ t)\ s$}
\RightLabel{beta\_conv}
\UnaryInfC{$\vdash (\lambda x.\ t)\ s = t[s/x]$}
\end{prooftree}

\begin{prooftree}
\def\labelSpacing{10 pt}
\AxiomC{$A_1 \vdash f_1 = f_2$}
\AxiomC{$A_2 \vdash t_1 = t_2$}
\RightLabel{mk\_comb\_rule}
\BinaryInfC{$A_1 \cup A_2 \vdash f_1\ t_1 = f_2\ t_2$}
\end{prooftree}

\begin{prooftree}
\def\labelSpacing{10 pt}
\AxiomC{$x$}
\AxiomC{$A \vdash t_1 = t_2$ [con $x$ non libera in $A$]}
\RightLabel{mk\_abs\_rule}
\BinaryInfC{$A \vdash (\lambda x.\ t_1)= (\lambda x.\ t_2)$}
\end{prooftree}

\begin{prooftree}
\def\labelSpacing{10 pt}
\AxiomC{$p$}
\RightLabel{assume\_rule}
\UnaryInfC{$\{p\}\vdash p$}
\end{prooftree}

\begin{prooftree}
\def\labelSpacing{10 pt}
\AxiomC{$p$}
\AxiomC{$A \vdash q$}
\RightLabel{disch\_rule}
\BinaryInfC{$A\backslash\{p\} \vdash p \Rightarrow q$}
\end{prooftree}

\begin{prooftree}
\def\labelSpacing{10 pt}
\AxiomC{$A \vdash p \Rightarrow q$}
\AxiomC{$A_2 \vdash p$}
\RightLabel{mp\_rule}
\BinaryInfC{$A_1 \cup A_2 \vdash q$}
\end{prooftree}

\begin{prooftree}
\def\labelSpacing{10 pt}
\AxiomC{$A_1 \vdash p \Leftrightarrow q$}
\AxiomC{$A_2 \vdash p$}
\RightLabel{eq\_mp\_rule}
\BinaryInfC{$A_1 \cup A_2 \vdash q$}
\end{prooftree}

\begin{prooftree}
\def\labelSpacing{10 pt}
\AxiomC{$[(tv_1,ty_1);(tv_2,ty_2);\ldots]$}
\AxiomC{$A \vdash p$}
\RightLabel{inst\_type\_rule}
\BinaryInfC{$A[ty_1/tv_1,ty_2/tv_2,\ldots] \vdash p[ty_1/tv_1,ty_2/tv_2,\ldots]$}
\end{prooftree}

\begin{prooftree}
\def\labelSpacing{10 pt}
\AxiomC{$[(x_1,t_1);(x_2,t_2);\ldots]$}
\AxiomC{$A \vdash p$}
\RightLabel{inst\_type\_rule}
\BinaryInfC{$A[t_1/x_1,t_2/x_2,\ldots] \vdash p[t_1/x_1,t_2/x_2,\ldots]$}
\end{prooftree}

\subsection{Conteggio delle inferenze primitive}

Sono matenuti due contatori per il numero di inferenze primitive effettuate dal sistema (indipendentemente dal fatto 
che esse derivino da chiamate dirette delle regole primitive di inferenza o da chiamate indirette attraverso regole 
di inferenza derivate). Il primo � un contatore assoluto che conta ogni passo primitivo eseguito, inclusi quelli fatti 
durante il build di HOL Zero. Questo pu� essere visualizzato attraverso il comando {\tt step\_total}:

\begin{hol}\begin{verbatim}

# step_total ();;
- : int = 68851

\end{verbatim}\end{hol}

Il secondo � un contatore relativo che conta i passi primitivi eseguiti dopo l'ultimo reset del contatore. Esso � resettato 
a zero alla fine del build del sistema. Esso pu� essere interrogato dal comando {\tt step\_counter} e resettato dal comando 
{\tt reset\_step\_counter}.

\begin{hol}\begin{verbatim}

# reset_step_counter ();;
- : unit = ()
# eval_add_conv `1+1`;;
- : thm = |- 1 + 1 = 2
# step_counter ();;
- : int = 12

\end{verbatim}\end{hol}

\section{Dichiarazione}

Il linguaggio HOL pu� essere esteso introducendo nuove costanti e costanti di tipo. Questo processo � chiamato {\it dichiarzione}.
Oltre ad estendere il linguaggio, una dichiarazione introduce anche le nuove costanti e le nuove costanti di tipo nella teoria 
(si veda la Sezione 4.5), ed � uno dei due meccanismi per costruire teorie (l'altro � l'asserzione, si veda la Sezione 4.4). Ci 
si riferisce genericamente alle costanti e alle costanti di tipo dichiarate come a {\it oggetti della teoria}.

Eseguire una dichiarazione ha un effetto immediato sulla sezione HOL Zero, dal momento che influiscono sui seguenti parsing e 
pretty printing delle quotation.

Si noti che in HOL Zero, non � possibile annullare una dichiarazione senza riavviare HOL Zero completamente. Questa limitazione 
permette di mantenere il kernel del linguaggio semplice.

\subsection{Dichiarazione di costante}

Le costanti sono introdotte usando il comando di dichiarazione \holtxt{new\_const}. Questo 
prende due argomenti: uno per il nome della costante, e uno per il suo tipo 
generico (si veda la Sezione 3.2.4). Oltre ad aggiungere la costante al linguaggio, il 
comando restituisce la nuova costante come un valore termine. Per esempio:

\begin{holboxed}
\begin{verbatim}
# new_const ("c", `:nat->bool`);;
- : term = `c`
\end{verbatim}
\end{holboxed}

Una volta che una costante � stata dichiarata, il suo tipo denerico pu� essere interrogato usando il 
comando \holtxt{get\_const\_gtype}. Per esempio:

\begin{holboxed}
\begin{verbatim}
# get_const_gtype "c";;
- : hol_type = `:nat->bool`
\end{verbatim}
\end{holboxed}

Un elenco di tutti i nomi di costante dichiarati e dei loro tipi generici � restituita dal 
comando \holtxt{get\_all\_consts}. Per esempio:

\begin{holboxed}
\begin{verbatim}
# get_all_consts ();;
- : (string * hol_type) list =
[("!", `:('a->bool)->bool`); ("*", `:nat->nat->nat`);
 ("+", `:nat->nat->nat`);
 ...]
\end{verbatim}
\end{holboxed}

In HOL Zero non � permesso l'overloading delle costanti con altre costanti, e 
se una dichiarazione � inserite per un nome di costante esistente ma con un tipo 
differente, allora la dichiarazione fallir�. Si noti che si pu� eseguire l'overloading 
delle costanti con altre entit�, incluse le variabili (si veda la Sezione 3.6.2).

Se la dichiarazione di una costante � ripetuta (cio� per lo stesso nome di costante e lo stesso 
tipo generico), allora questa � riconosciuta come una ``ri-dichiarazione benigna'', e non causa 
fallimento (ma non esegue alcun cambiamento sul linguaggio). La ri-dichiarazione benigna � 
permessa per permettere di reinserire uno script di dimostrazione nella stessa sessione HOL Zero 
senza causare fallimento.



   % \appendix
   %    \input{matembackground}
   %    \input{lightOcaml}
   %    \input{parsing_printing}

   % \bibliographystyle{apalike}	% or "unsrt", "alpha", "abbrv", etc.
   % \bibliography{biblio}		% use data in file "biblio.bib"
   % \newcommand{\bibtitle}{Bibliografia}
   % \addcontentsline{toc}{chapter}{\bibtitle}
	
	\printindex
% \newcommand{\glossario}{Indice analitico}
% \addcontentsline{toc}{chapter}{\glossario}

\end{document}
